\documentclass[systemskiss/skiss.tex]{subfiles}
\begin{document}
\section{Mjukvara}
\subsection{Klient}
Klienten på den bärbara datorn ska sköta användargränssnittet. Klienten ska
grafiskt rita kartan samt positionen av taxins bil. Man ska kunna mata in
reglerparemeter och fjärrstyra antingen via det grafiska gränssnittet eller via
terminalen. Klienten kan även ansvara för att räkna ut den kortaste vägen,
därefter klienten i samband med överföring av karta även skicka den
kortaste vägen.

Klienten kan exempelvis skrivas i Python, det har bibliotek som TKinter för
GUI. Bibliotek som matplotlib kan dessutom vara användbara för att plotta
grafer av värdena som skickas från bilen.

\subsection{Kommunikationsmodul}
Kommunikationsmodulen som består av en Raspberry Pi är den centrala enheten och
den är även taxins beslutsenhet. Programmet som körs ska ta in alla datavärden
från kopplade moduler och agera utefter värdena. Programmet ska skrivas i C
eller Python där huvudprogrammet är en while-loop. Programmet ska även bearbeta
bilddata med hjälp av biblioteket OpenCV. Utifrån bilddatan byggs skelett för
att känna igen vägfiler och hinder samt färgerna på linjerna.

\subsection{Sensormodul}
Mjukvaran på sensormodulens mikrokontroller kommer att ta emot och bearbeta
värden från sensorerna. Den kommer att sköta kommunikationen till
kommunikationsmodulen. Mjukvaran lär skrivas i C.


\subsection{Styrmodul}
Mjukvaran på styrmodulen kommer att ta emot beslut från kommunikationmodulen och
utifrån besluten skicka lämpliga värden till drivmotorn och svängmotorn.
Mjukvaran ansvarar även för att skicka värden till LCD-skärmen dessutom ska 
mikrokontrollern skicka värdena till kommunikationsmodulen eftersom
kommunikationsmodulen ska kunna räkna ut hastighet, svängningsradie och
därefter även räkna ut var den befinner sig i banan. Mjukvaran lär skrivas i C.
 
\end{document}
