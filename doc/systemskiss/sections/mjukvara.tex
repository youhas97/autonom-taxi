\documentclass[systemskiss/skiss.tex]{subfiles}
\begin{document}
\section{Mjukvara}
\subsection{Klient}
Klienten på den bärbara datorn ska sköta användargränssnittet. Klienten ska
grafiskt rita kartan samt positionen. Man ska kunna mata in
reglerparemeter och fjärrstyra antingen via det grafiska gränssnittet eller via
terminalen. Klienten ska även ansvara för att räkna ut den kortaste vägen,
därefter kan klienten i samband med överföring av karta även skicka den
kortaste vägen. För att undvika krångel ska klienten vara skrivet i ett språk
och python verkar gruppen vara överens om ty det är enkelt och har en hel del
bibliotek som kan användas till detta ändamålet. För det grafiska gränssnittet
kan man använda biblioteket tkinter. Vi kan även använda scipy för att plotta
grafer av värdena som skickas från bilen

\subsection{Kommunikationsmodul}
Kommunikationsmodulen som består av en raspberry pi är den centrala enheten och
den är även beslutsenheten. Programmet som körs ska ta in alla datavärden från
kopplade moduler och agera utefter värdena. Programmet ska skrivas i C där
huvudprogrammet är en while-loop. Programmet ska även bearbeta bilddata med
hjälp av biblioteket openCV. Utifrån bilddatan ska man bygga skelett och känna
igen vägfler, hinder samt färgerna på linjerna.  


\subsection{Mikrokontroller}
Mjukvaran till mikrokontrollerna behöver flashas, antagligen via JTAG. Det vore
smidigt ifall vi kunde flasha dessa mikrokontroller från kommunikationsmodulen
istället för att behöva flasha de var för sig med en dator. Vi ska välja
mikrokontroller som har en kompilator för C, t.ex avrgcc. 

\subsection{Testning}
Under utvecklingen kommer saker att testas, bland annat modultester för att
tidigt upptäcka mjukvaru- eller hårdvarufel. Om det lönar sig med att
automatisera tester så kommer vi göra det. I slutfasen av projektet kommer en
hel del tid gå åt till integrerande tester och en generell test över hela
systemet.

\end{document}
