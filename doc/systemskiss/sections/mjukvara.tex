\documentclass[systemskiss/skiss.tex]{subfiles}
\begin{document}
\section{Mjukvara}
\subsection{Klient}
Klienten på den bärbara datorn ska sköta användargränssnittet och kommunicera
med kommunikationsmodulen. Klienten ska grafiskt rita kartan samt positionen av
taxins bil. Reglerparemeter ska kunna matas in och bilen ska kunna fjärrstyras
via gränssnittet. Klienten kan även ansvara för att räkna ut den kortaste
vägen, därefter i samband med överföring av karta dessutom skicka den kortaste
vägen.

Klienten kan exempelvis skrivas i Python, det har bibliotek som TKinter för
GUI. Bibliotek som matplotlib kan dessutom vara användbara för att plotta
grafer av värdena som skickas från bilen.

\subsection{Kommunikationsmodul}
Kommunikationsmodulen som består av en Raspberry Pi är den centrala enheten och
den är även taxins beslutsenhet. Programmet som körs ska ta in alla
sensorvärden från sensormodulen och kameran, och därefter agera utefter
värdena. Programmet ska skrivas i C eller Python. Programmet ska även bearbeta
bilddata med hjälp av biblioteket OpenCV. Utifrån bilddatan byggs skelett för
att känna igen vägfiler och hinder samt färgerna på linjerna.

\subsection{Sensormodul}
Mjukvaran på sensormodulens mikrokontroller kommer att ta emot och bearbeta
värden från modulens sensorer. Den kommer att sköta kommunikationen till
kommunikationsmodulen och lär skrivas i C.


\subsection{Styrmodul}
Mjukvaran på styrmodulen kommer att ta emot beslut från kommunikationmodulen och
utifrån besluten skicka lämpliga värden till drivmotorn och svängmotorn.
Mjukvaran ansvarar även för att skicka värden till LCD-skärmen dessutom ska 
mikrokontrollern skicka värdena till kommunikationsmodulen eftersom
kommunikationsmodulen ska kunna räkna ut hastighet, svängningsradie och
därefter även räkna ut var den befinner sig i banan. Mjukvaran lär skrivas i C.
 
\end{document}
