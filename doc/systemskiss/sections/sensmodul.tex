\documentclass[systemskiss/skiss.tex]{subfiles}

\begin{document}
\section{Sensormodul}
Sensormodulen innehåller alla de sensorer som bilen behöver för att kunna mäta
avstånd i sin omgivning. Den är direkt kopplad till kommunikationsmodulen via
en databuss.
\subsection{Översiktlig beskrivning av modulen}
\begin{figure}[h]
    \centering
    \includegraphics[width=0.6\linewidth]{systemskiss/figures/sensormodul.pdf}
    \caption{Övergripande bild över sensormodulen}
    \label{fig:sensorskiss}
\end{figure}
\noindent Sensorerna ska vara kopplade till olika brusfilter som ska filtrera
bort störningar av sensorvärdena. Den bakre sensor är tänkt att vara en
IR-avståndsmätare. Bilen kommer använda denna sensor till att upptäcka när den
har kört förbi hinder. Avståndsmätare där framme ska kunna mäta avstånd till
hinder. Det kommer vara antingen en ultraljudssensor eller en IR-sensor som
åtminstone kan som minst mäta 5 cm och iaf 30 cm framåt eller längre. Bilen
kommer även vara utrustad med någon form av ordometer som ska mäta hur långt
bilen har kört i banan. Den ska placeras vid något av bilens däck och mäta
antalet varv däcket snurrar. Sensorvärden kommer skickas via en databuss till
modulens processor. En LCD-skärm ska kopplas till mikrokontrollern eller direkt
till bussen. På LCD-skärmen ska sensorvärden visas för att förenkla felsökning
av modulen. Modulen ska ha en direkt koppling till kommunikationsmodulen via en
databuss. Figur \ref{fig:sensorskiss} visar en grov bild över hur modulen ska
se ut.


\end{document}

