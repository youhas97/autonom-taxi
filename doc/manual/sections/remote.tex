\documentclass[manual/man.tex]{subfiles}
\section{Fjärrstyrning}
För att styra bilen måste man först initiera en trådlös länk mellan bilens
kommunikationsmodul och datorn man använder. Detta görs genom att starta en
server via kommandot make comm.remote. När servern är på kan man ansluta till
bilen via det grafiska användargränssnittet.
    
\subsection{Grafiskt Användargränssnitt}
Det finns ett grafiskt användargränssnitt som man kan använda för att koppla
upp sig till bilen. För att starta gränssnittet kör man filen main.py.

\subsubsection{Anslutning}
För att ansluta till den startade servern navigerar man till
System-Server-Connect. Efter att man tryckt på Connect kommer en ruta upp där
man kan skriva in en IP-adress. IP-adressen som ska skrivas in är samma
IP-adress bilen är uppkopplad till. För att bekräfta sitt val av IP måste
enter-knappen på tangentbordet tryckas ner. Ett meddelande skrivs ut i
terminalen efter att man trycker på enter, med ett felmeddelande ifall klienten
inte lyckas anslutas till bilen

\subsubsection{Körning}
Efter att ha anslutit till bilen kan man välja ifall man vill styra bilen
manuellt eller ifall man vill att bilen skall köra autonomn. För att välja
detta navigerar man till Driving-Manual respektive Driving-Auto.

För att ändra reglerparametrarna navigerar man till Driving-Set KP/KD. En ny
ruta kommer upp där man kan ange värden för parametrarna man vill ändra. För
att bekräfta sitt val av KP/KD måste man, precis som tidigare, trycka ner
enter-knappen på tangentbordet.
