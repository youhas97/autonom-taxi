\documentclass[kravspec/krav.tex]{subfiles}

\begin{document}

\section{Översikt av systemet}
Här beskrivs produkten övergripande där generella krav ställs på produkten och dess avgränsningar.
\begin{figure}[h]
    \centering
    \includegraphics[width=0.6\linewidth]{kravspec/figures/overview-schema.pdf}
    \caption{Övergripande bild över systemet och dess moduler.}
    \label{fig:overview}
\end{figure}

\subsection{Grov beskrivning av produkten}
Produkten är en bil med fyra hjul som ska med hjälp av ett känt vägnät ta sig
till en passagerare och skjutsa passageraren till önskad destination. Denna
taxibil ska kunna åka framåt, bakåt, svänga vänster och höger. Bilen kommer ha
en kamera som tar bilder från en riktning eller flera.  Bland annat ska
man kunna initiera en bluetooth-länk mellan bilen och en dator som har stöd för
bluetooth. Bilen ska ha ett läge där man kan fjärrstyra bilen och ett läge där
bilen ska köra autonomt i vägnätet samt undvika hinder.

\subsection{Produktkomponenter}
Lämplig hårdvara med kompletterande mjukvara kommer finnas i den kompletta
produkten. Bland annat ska även tekniskt dokumenation ingå i produkten.

\subsection{Beroenden till andra system}
Fjärstyrning av bilen skall vara beroende av en dator som har stöd för bluetooth.
Autonoma läget aktiveras från användargränsnittet tillgänglig på datorn med
bluetooth. Kameran kommer vara beroende till en mikroprocessor som ingår i systemet.

\subsection{Ingående delsystem}
Systemet kommer ska bestå av delsystem enligt figur
\ref{fig:overview}. Sensormodulen ska hämta data om omgivningen, bland annat en sensor som mäter avstånd
till objekt i omgivning och tejpföljare som sensorer. Lämplig data av omgivning
skickas till kommunikationsmodell. Produkten har även en styrmodul som ser till
att bilen kan styras beroende på data från kommunikationsmodul.

\subsection{Avgränsningar}
Små väghinder som låga grova vägkanter och gropar tas ej hänsyn till. Vägen förväntas bestå av hinder som är
fördefinierade i banspecifikationen. Vägen antas vara plan utan gropar och markerad med tejp. Parkeringsfickan
förväntas vara markerad med en specifik färg som skiljer sig från färgen på vägmarkeringarna. Dessutom
ställs inga krav på att kameran ska ingå i en specifik modul, om den behöver vara beroende till kommunikationensenheten så placeras den i den modulen. Om det behövs så får kameran vara en egen modul, men ett krav på detta ställs inte.


\subsection{Generella krav på hela systemet}

\begin{reqlist}
    \req{Taxin ska kunna skicka aktuell mätdata till den bärbara datorn.
    Mätdata inkluderar avstånd till vägkant, avstånd till hinder, avlagd
    sträcka, styrbeslut och motorernas styrning}
    \req{Taxin skall via fjärrstyrning kunna köra framåt, bakåt, stanna och
    svänga vänster eller höger.}
    \req{Taxin skall ha en kamera som tar bilder åtminstone åt en riktning }
    \req{Under autonom körning skall taxin navigera vägnätet enligt
    högertrafik.}
    \req{Under autonom körning skall taxin ej köra in i hinder.}
    \reqspec{original}{2}{Om ett hinder blockerar taxins körfält under autonom
    körning skall taxibilen köra runt objektet genom att köra i det andra
    körfältet.}
    \req{Under autonom körning skall taxin navigera genom vägnätet från en
    godtycklig position till en godtycklig given destination.}
    \reqspec{original}{3}{Bilen skall aktivera blinkers vid svängning samt
    hämtning och avlämning.}
\end{reqlist}

\end{document}
