\documentclass[kravspec/krav.tex]{subfiles}

\begin{document}
\section{Introduktion}
TODO introtext

\subsection{Parter}
Beställare av systemet är Mattias Krysander.  Utvecklingsgruppen består av
Jakob Arvidsson, Juan Basaez, Johan Can, Dennis Derecichei, Emir
Hadzisalihovic, Yousef Hashem, Noah Hellman.  Handledare till
utvecklingsgruppen är *yash*

\subsection{Syfte och Mål}
Målet är att konstruera en autonom taxibil som ska kunna köra i en bana och
plocka upp samt lämna av passagerare. I banan kan det förekomma hinder av olika
slag. Syftet med uppgiften är att fördjupa våra kunskaper inom elektronik,
programmering och framför allt samarbete inom en projektgrupp.

\subsection{Användning}
Konstruktionen ska användas och sättas på prov i en tävling mellan alla grupper som tillhör samma projekttyp. Detta för att visa att konstruktionen fungerar som den ska samt att kraven är uppfyllda.

\subsection{Bakgrundsinformation}
Projektet startades med att en beställare beskrev ett system och ställde krav
utifrån funktionen. Dessa krav måste vara tydliga för att minimera missförstånd
mellan beställare och utvecklare. Kraven antecknas, specificeras och
sammanställs i rapporten nedan.

\subsection{Definitioner}
\begin{labeling}{långt namn}
    \item[Taxin] Namnet som används på den radiostyrda bilen där systemet är
    implementerat.
    \item[Modul] Är en oberoende del i konstruktionen som består av minst en
    processor. Moduler ska vara lätta att byta ut mot andra moduler.
    \item[Vägnät] Den tvåfiliga bana som taxin ska köra på. Den består av vita
    vägar med svarta sidmarkeringar.
    \item[Hinder] Något som är placerat på vägen och förhindrar framkomsten för
    Taxin när den kör på vägnätet.
    \item[Autonom körning] Ett visst läge i konstruktionen där Taxin är menad
    att transportera passagerare.
    \item[Bärbara datorn] Den dator som är ansluten till Taxin via Bluetooth.
    \item[Fjärrstyrning] Styrning av Taxin ifrån den bärbara datorn. 
\end{labeling}

\end{document}
