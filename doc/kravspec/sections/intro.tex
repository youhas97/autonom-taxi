\documentclass[kravspec/krav.tex]{subfiles}

\begin{document}
\section{Inledning}
Den här kravspecifikationen svarar på alla möjliga frågor angående konstruktionen av den tidigare nämnt autonoma taxibilen. På följande sidorna, under olika tydliga punkter, förklaras alla krav som produkten ska uppfylla, allt som ska användas vid konstruktionen och hur alla komponenter samt mjukvara ska vara kopplade vid användningen.

Här hittas alla förklaringar av hur produkten ska konstrueras så att beställaren av produkten kan i princip komma överens med gruppen som ska vara ansvarig för konstruktionen och på det sättet bekräfta att produkten ska uppfylla hens krav.


\subsection{Parter}
Beställare av systemet är
\begin{list}{}
<<<<<<< HEAD
       \item Mattias Krysander  \newline    
\end{list}

\noindent
Utvecklingsgruppen består av:

\begin{list}{}
       \item Jakob Arvidsson
       \item Juan Basaez
       \item Johan Can
       \item Dennis Derecichei
       \item Emir Hadzisalihovic 
       \item Yousef Hashem
       \item Noah Hellman \newline
\end{list}
=======
    \item \bfseries Mattias Krysander.
\end{list}\vspace{4mm}

\noindent
Utvecklingsgruppen består av
{\bfseries\begin{list}{}
   \item Jakob Arvidsson,
   \item Juan Basaez,
   \item Johan Can,
   \item Dennis Derecichei,
   \item Emir Hadzisalihovic,
   \item Yousef Hashem,
   \item Noah Hellman.
\end{list}}\vspace{4mm}
>>>>>>> f2c35455bb5f848bcc4dbe1868d87d4c4a755647

\noindent
Handledare till utvecklingsgruppen är
\begin{list}{}	
<<<<<<< HEAD
	\item *yashimilimimi* \newline
\end{list}
=======
	\item \bfseries TBD.
\end{list}\vspace{4mm}
>>>>>>> f2c35455bb5f848bcc4dbe1868d87d4c4a755647

\subsection{Syfte och Mål}
Målet är att konstruera en autonom taxibil som ska kunna köra i en bana och
plocka upp samt lämna av passagerare. I banan kan det förekomma hinder av olika
slag. Syftet med uppgiften är att fördjupa våra kunskaper inom elektronik,
programmering och framför allt samarbete inom en projektgrupp.

\subsection{Användning}
Konstruktionen ska användas och sättas på prov i en tävling mellan alla grupper
som tillhör samma projekttyp. Detta för att visa att konstruktionen fungerar
som den ska samt att kraven är uppfyllda.

\subsection{Bakgrundsinformation}
Projektet startades med att en beställare beskrev ett system och ställde krav
utifrån funktionen. Dessa krav måste vara tydliga för att minimera missförstånd
mellan beställare och utvecklare. Kraven antecknas, specificeras och
sammanställs i rapporten nedan.

\subsection{Definitioner}
\begin{labeling}{långt namn}
    \item[Taxin] Namnet som används på den radiostyrda bilen där systemet är
    implementerat.
    \item[Modul] Är en oberoende del i konstruktionen som består av minst en
    processor. Moduler ska vara lätta att byta ut mot andra moduler.
    \item[Vägnät] Den tvåfiliga bana som taxin ska köra på. Den består av vita
    vägar med svarta sidmarkeringar.
    \item[Hinder] Något som är placerat på vägen och förhindrar framkomsten för
    Taxin när den kör på vägnätet.
    \item[Autonom körning] Ett visst läge i konstruktionen där Taxin är menad
    att transportera passagerare.
    \item[Bärbara datorn] Den dator som är ansluten till Taxin via Bluetooth.
    \item[Fjärrstyrning] Styrning av Taxin ifrån den bärbara datorn. 
\end{labeling}

\end{document}
