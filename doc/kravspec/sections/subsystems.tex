\documentclass[kravspec/krav.tex]{subfiles}

\begin{document}
\section{Kommunikationsmodul}

\section{Styrmodul}
\begin{figure}[h]
    \centering
    \includegraphics[width=0.6\linewidth]{kravspec/figures/styrmodul.pdf}
    \caption{Styrmodul}
Styrmodulen har som uppgift att ställa hjulen i rätt position samt att Taxin håller rätt hastighet. Styrmodulen är kopplad till kommunikationsmodulen ifrån den får information om hur Taxins hjul ska bete sig. Styrmodulen består av en kontroller, en drivmotor och en styrmotor. Styrmotorns uppgift är att hjulen ska vara i rätt position så att TAxin åker i önskad riktning. Drivmotorn är den motor som gör att bilen får en hastighet. Alltså köra fram eller bakåt.
    \label{fig:styrmodul}
\end{figure}

\section{Sensormodul}
\subsection{Beskrivning}
Sensormodulen består av enhetens alla sensorer och en microprocessor. Dess
uppgift är att hantera information från de olika sensorerna och även
ut
\subsection{Gränssnitt}
\subsection{Krav}

\end{document}
