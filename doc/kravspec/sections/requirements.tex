\documentclass[kravspec/krav.tex]{subfiles}

\begin{document}
\clearpage
\section{Prestandakrav}
Nedan följer de prestandakrav taxibilen bör uppnå vid avslutat projekt.

\begin{reqlist}
    \req{Bilen ska under fyra försök klara av att utföra ett uppdrag, som
    specificerat i banspecifikationen, åtminstone tre gånger.
    Banspecifikationen skickas in separat till beställaren.}
    \req{Bilen ska endast använda kortaste vägen under uppdraget.}
    \req{Bilen skall vid hämtning och avlämning stanna åtminstone 5 sekunder.}
    \reqspec{original}{2}{Bilen ska klara uppdraget fyra gånger på fyra försök.}
\end{reqlist}

\section{Krav på utökning}
Nedan följer de krav som berör möjligheten att utöka enheten efter att
produkten är levererad.

\begin{reqlist}
    \req{Enheten ska bestå av separata moduler som kommunicerar via ett
    gränssnitt för att tillåta smidiga utbyten av moduler. Gränssnittet
    specificeras i det tekniska dokumentet.}
\end{reqlist}

\section{Resurser}
Projektgruppen skall hålla sig inom följande resurskrav.

\begin{reqlist}
    \req{Projektgruppens medlemmar skall spendera 160 timmar vardera på
    projektet efter beslutspunkt 2.}
\end{reqlist}

\section{Leveranskrav}
Följande leveranskrav skall uppnås.

\begin{reqlist}
    \req{Tidsrapporter som redovisar föregående veckas arbetad tid ska
    levereras vid varje måndag från och med vecka 46 till och med vecka 51
    2018.}
    \req{De första versionerna av projektplanen, tidplanen och systemskissen
    ska levereras till beställaren senast 27 september 2018.}
    \req{De slutgiltiga versionerna av projektplanen, tidplanen och en
    systemskissen ska levereras till beställaren senast 4 oktober 2018.}
    \req{Den första versionen av designspecifikationen skall levereras till
    handledaren senast 6 november 2018.}
    \req{Den slutgiltiga versionen av designspecifikationen skall levereras
    till handledaren senast 9 november 2018.}
    \req{Verifiering för kraven på beslutspunkt 5 (LIPS-modellen) ska vara
    verifierade senast dagen innan redovisning av projektet, d.v.s. vecka 51
    (17-21 December) 2018.}
    \req{En efterstudie på projektet skall göras och lämnas in.}
    \req{Den använda hårdvaran för taxin skall återlämnas.}
\end{reqlist}

\section{Dokumentationskrav}
\label{sec:doc}
Följande dokumentationskrav skall uppnås.
\begin{reqlist}
    \req{All dokumentation som lämnas in ska utgå från LIPS-modellen}
\end{reqlist}

\section{Dokumentation}
\label{sec:doc}
Nedan följer en lista av de dokument som skall medfölja produkten vid leverans.
{\renewcommand{\arraystretch}{1.6}
\begin{longtable}{p{4.5cm}p{1.5cm}p{5cm}p{2cm}p{1.2cm}}
    \bfseries Dokument &
    \bfseries Språk &
    \bfseries Syfte &
    \bfseries Målgrupp &
    \bfseries Format \\\hline
    Teknisk dokumentation &
    Svenska &
    Ge konstruktionsunderlag, samt dokumentation för underhåll och
    felsökning. &
    Beställare &
    PDF
    \\
    Användarhandledning &
    Svenska &
    Ge beskrivning av produktens användning. &
    Beställare, användare &
    PDF
    \\
    Designspecifikation &
    Svenska &
    Designspecifikationen ska spegla kravspecifikationen.
    Här beskrivs implementering av hårdvaran samt mjukvaran till systemet. &
    Underhållare, utvecklare &
    PDF
    \\
    Projektplan &
    Svenska &
    Här beskrivs hur projektet kommer att utföras
    och vilka strategier som ska tillämpas. &
    ingen aning &
    PDF
    \\
    Tidplan &
    Svenska &
    Här delar man upp tidsresurserna i aktiviter och planerar
    vilka dagar en viss person ska lägga ner på en viss aktivitet. &
    Löneavdelning, projektledaren &
    PDF
    \\
    Systemskiss &
    Svenska &
    En grov skiss över implementering/implementation av systemet &
    Beställare, utvecklare &
    PDF
    \\
    
    \endhead
\end{longtable}
}

\end{document}
