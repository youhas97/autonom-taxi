\documentclass[kravspec/krav.tex]{subfiles}

\begin{document}
\section{Prestandakrav}
Nedan följer de prestandakrav taxibilen bör uppnå vid avslutat projekt.
Prestandakraven har vardera en prioritet där 1 mostvarar något som skall vara
klart och 2 motsvarar att det bör vara klart.

\begin{reqlist}
    \req{Taxin ska under fyra försök klara av uppdraget åtminstone tre gånger.}
    \req{Taxin ska endast använda kortaste vägen under uppdraget.}
    \reqspec{original}{2}{Taxin ska klara uppdraget fyra gånger på fyra försök.}
\end{reqlist}

\section{Krav på utökning}
Nedan följer de krav som berör möjligheten att utöka enheten efter att
produkten är levererad.

\begin{reqlist}
    \req{Enheten ska bestå av separata moduler som kommunicerar via ett
    specificerat gränssnitt.}
\end{reqlist}

\section{Ekonomi}
Projektgruppen skall hålla sig inom följande ekonomiska krav.

\begin{reqlist}
    \req{Projektgruppens medlemmar skall tillsammans spendera maximalt 960
    timmar på projektet efter beslutspunkt 2.}
\end{reqlist}

\section{Leveranskrav}
Vid leverans av produkten skall följande krav uppnås.

\begin{reqlist}
    \req{En produkt som följer tidigare specificerade krav skall levereras.}
    \req{Dokumentation som specificerad i sektion \ref{sec:doc} skall
    levereras.}
\end{reqlist}

\section{Dokumentation}
\label{sec:doc}
Nedan följer en lista av de dokument som skall medfölja produkten vid leverans.
\begin{longtable}{p{4.5cm}p{1.5cm}p{5cm}p{2cm}p{1.2cm}}
    \bfseries Dokument &
    \bfseries Språk &
    \bfseries Syfte &
    \bfseries Målgrupp &
    \bfseries Format \\\hline
    Teknisk dokumentation & Svenska & Ge konstruktionsunderlag, samt dokumentation för
    underhåll och felsökning. & Användare & PDF \\
    Användarhandledning & Svenska & Ge beskrivning av produktens användning. & Användare & PDF
    % designspec, användarmanual
    \endhead
\end{longtable}

\end{document}
