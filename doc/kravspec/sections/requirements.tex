\documentclass[kravspec/krav.tex]{subfiles}

\newcounter{reqcat} % nummer för varje kravkategori
\newcounter{reqitem} % individuella kravnummer

\newenvironment{reqlist}
{
    \stepcounter{reqcat}
    \begin{tabular}{|l|l|p{10cm}|l|}
    \hline
    \bfseries nr &
    \bfseries modif. &
    \bfseries krav &
    \bfseries prioritet \\
}{
    \hline
    \end{tabular}
}
\newcommand{\req}[3]{
    \hline
    \stepcounter{reqitem}\thereqcat.\thereqitem & #1 & #3 & #2 \\
}

\begin{document}
\section{Krav}
TODO: introtext

\subsection{Prestandakrav}
TODO: introtext
\begin{reqlist}
\req{original}{1}{
    Taxin skall åtminstone varje sekund skicka aktuell mätdata till den bärbara
    datorn. Mätdata inkluderar avstånd till vägkant, avstånd till hinder, TODO
}
\req{original}{1}{
    Taxin skall via fjärrstyrning kunna köra framåt, bakåt och svänga vänster
    eller höger.
}
\req{original}{1}{
    Taxin ska processera åtminstone en bild per sekund.
}
\req{original}{1}{
    Under autonom körning skall taxin navigera vägnätet enligt högertrafik.
}
\req{original}{1}{
    Under autonom körning skall taxin ej köra in i hinder. (NOTE: är hinder
    permanenta? isf köra om)
}
\req{original}{2}{
    Om ett hinder blockerar taxins körfält under autonom körning skall
    taxibilen köra runt objektet genom att köra i det andra körfältet.
}
\req{original}{1}{
    Under autonom körning skall taxin navigera genom vägnätet från en
    godtycklig position till en godtycklig given destination.
}
\end{reqlist}

\subsection{Krav på utökning}
TODO: introtext

\begin{reqlist}
\req{original}{1}{
    Enheten ska bestå av separata moduler som kommunicerar via ett specificerat
    gränssnitt.
}
\end{reqlist}

\end{document}
