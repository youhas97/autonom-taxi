\documentclass[kravspec/krav.tex]{subfiles}

\begin{document}

\vspace*{4cm}
\renewcommand{\abstractname}{\Large Sammanfattning}
\thispagestyle{empty}
\begin{abstract}
    \vspace*{1cm} \noindent
    Det här dokumentet är en kravspecifikation angående konstruktionen av en
    autonom taxibil vilken ska konstrueras under ett projekt för kursen TSEA29,
    ``Konstruktion med mikrodatorer'', vid Linköpings universitet. Projektet
    kommer att utföras av sju studenter under Hösttermin 2018. 

    Den autonoma taxibilen ska köra, i ett känt vägnät, från en punkt till en
    annan för att simulera hämtningen och avlämningen av en passagerare via den
    kortaste möjliga vägen (utan att kollidera med hinder under resan).  Bilen
    ska bestå av tre moduler; en kommunikationsmodul, en styrmodul och en
    sensormodul. Moduler ska enkelt kunna ersättas vid behov.  Bilen ska
    innehålla flera olika sensorer och en kamera som möjliggör autonom
    körningen. En användare skall från en bärbar dator kunna styra bilen så att
    den kör framåt, backar, svänger eller stannar.
\end{abstract}
\newpage
\end{document}
