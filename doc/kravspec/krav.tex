\documentclass[12pt]{article}

% svensk encoding, svenska titlar
\usepackage[utf8]{inputenc}
\usepackage[swedish]{babel}

% figurer
\usepackage{graphicx} % pdf-import
\usepackage{float} % Figure H option
\usepackage{longtable} % splitta tabeller vid sidbrytning

% separera filer, subfile kommando
\usepackage{subfiles}

% sidomargins
\usepackage[total={16cm, 21cm}]{geometry}

% labeling env för definitioner
\usepackage{scrextend}
\addtokomafont{labelinglabel}{\bfseries}

% highlighting for links
\usepackage[colorlinks=false,urlcolor=blue,linkcolor=black,citecolor=blue,filecolor=blue]{hyperref}
\hypersetup{breaklinks=true}
\urlstyle{same}

% page header and foot
\usepackage{fancyhdr}
\pagestyle{fancy}
\renewcommand{\headrulewidth}{0.2pt}
\renewcommand{\footrulewidth}{0.2pt}
\rhead{\sc \LIPSdatum}
\chead{\textsl{\VARprojekttitel}}
\lhead{\textbf{\sc\footnotesize\leftmark}}
\lfoot{\textbf{\sc TSEA29}\\ \LIPSdokumenttyp}
\rfoot{\textbf{\sc Grupp 2}\\\VARdokumentansvarig}

%tabell för krav
\newcounter{reqcat} % nummer för varje kravkategori
\newcounter{reqitem} % individuella kravnummer

\newenvironment{reqlist}
{
    \setlength{\tabcolsep}{12pt}
    \renewcommand{\arraystretch}{1.6}
    \stepcounter{reqcat}
    \begin{longtable}{p{8mm}p{13mm}p{9cm}p{18mm}}
        \bfseries Nr &
        \bfseries Modif. &
        \bfseries Krav &
        \bfseries Prioritet \\\hline
        \endhead
}{
    \end{longtable}
}
\newcommand{\reqspec}[3]{
    \stepcounter{reqitem}\thereqcat.\thereqitem & #1 & #3 & #2 \\
}
\newcommand{\req}[1]{\reqspec{original}{1}{#1}}

%tabell för dokumenthistorik
\newenvironment{dokumenthistorik}{%
\begin{center}
  Dokumenthistorik\\[1ex]
  \begin{small}
    \begin{tabular}{|l|l|p{60mm}|l|l|}
      \hline
      \textbf{Version} & \textbf{Datum} & \textbf{Utförda förändringar} & \textbf{Utförda av} & \textbf{Granskad} \\
      }%
      {%
      \hline
    \end{tabular}
  \end{small}
\end{center}
}
%argument version, datum, utförda förändringar, utförda av, granskad
\newcommand{\versioninfo}[5]{\hline {#1} & {#2} & {#3} & {#4} & {#5} \\}


\newcommand{\LIPSdokumenttyp}{Designspec}
\newcommand{\LIPSversion}{TODO version}
\newcommand{\LIPSdatum}{TODO datum}
\newcommand{\LIPSgranskare}{TODO granskare}
\newcommand{\LIPSgodkannare}{TODO godkännare}
\newcommand{\LIPSgranskatdatum}{TODO gr datum}
\newcommand{\LIPSgodkantdatum}{TODO godk datum}
\newcommand{\docdir}{designspec}
\newcommand{\sections}{\docdir/sections}

\newcommand{\distFModel}{GP2D120}
\newcommand{\distFMin}{20}
\newcommand{\distFMax}{150}
\newcommand{\distSModel}{GP2Y0A02YK}
\newcommand{\distSMin}{4}
\newcommand{\distSMax}{30}


\newcommand{\LIPSdokumenttyp}{Kravspecifikation}
\newcommand{\LIPSversion}{0.2}
\newcommand{\LIPSdatum}{11 september 2018}
\newcommand{\LIPSgranskare}{}
\newcommand{\LIPSgodkannare}{}
\newcommand{\LIPSgranskatdatum}{}
\newcommand{\LIPSgodkantdatum}{}


\begin{document}
\subfile{kravspec/sections/frontpage}
\vspace*{1cm}
\renewcommand{\abstractname}{\Large Sammanfattning}
\begin{abstract}
\vspace*{1cm}

Det här dokumentet är en kravspecifikation angående konstruktionen av en autonom taxibil vilken ska konstrueras under ett projekt för kursen "Konstruktion med mikrodatorer" (TSEA29) vid Liu Universitet. Projekt som kommer att utföras för sju studenter under Hösttermin 2018. 

Den autonoma taxibilen ska köra, i ett känt vägnät, från en punkt till en annan för att simulera hämtningen och avlämningen av en passagerare via den kortaste möjliga vägen (utan att kollidera med hinder under resan).
Bilen ska räkna med tre moduler, kommunikationsmodul via blåtand, styrmodul och sensormodul; moduler ska enkelt kunna växlas vid behov eller önska.
Bilen innehåller många sensorer och en kamera, enheter som kommer att, tillåta den autonoma körningen samt fjärrövervakningen från en bärbar dator; men bilen ska också kunna styras manuellt från en bärbar där användaren ska kunna styra bilen så att den svänger (höger, vänster), backar eller stannar.


\end{abstract}
\newpage
\tableofcontents\newpage
\begin{dokumenthistorik}
    \versioninfo{0.1}{2018-09-10}{Första utkast.}{Grupp 2}{Hela
    gruppen}
    \versioninfo{0.2}{2018-09-11}{Andra utkast.}{Grupp 2}{Hela
    gruppen}
\end{dokumenthistorik}
\newpage
\subfile{kravspec/sections/intro}\clearpage
\subfile{kravspec/sections/overview}\clearpage
\subfile{kravspec/sections/subsystems}
\subfile{kravspec/sections/requirements}
\end{document}
