\documentclass[tekniskrapport/tech.tex]{subfiles}

\begin{document}

\section{Styrmodul}
Styrmodulen är en modul som har i uppgift att ta emot kommandon och data ifrån
kommunikationsmodulen och reglera bilens drivmotor och svängservo utifrån
dessa.

\subsection{Funktion}
Styrmodulens syfte är att hantera den direkta kontrollen av styrreglagen för
taxins motor. Styrmodulen reglerar drivmotorn och svängservo utifrån
felvärden och reglerkonstanter som tas emot från kommunikationsmodulen.

\subsection{Hårdvaruimplementation}
Styrmodulen består av en mikrokontroller, taxins motor, en servo samt en LCD för
felsökning. Mikrokontrollern är klockad av en kristalloscillator på 16 MHz. Ett
detaljerat kretsschema finns i bilaga \ref{cdiag:ctrl}.

\subsubsection{Budget}
Utöver motorn och servon som redan sitter monterade på chassit använder
styrmodulen följande komponenter.
\begin{itemize}
	\item \textbf{\modMicrocontroller} Modulens microprocessor. 
    \item \textbf{\modJtag} Debugger för programmering och felsökning med
        microprocessorn. Även till för att flasha programmet som
        microprocessorn ska köra.
    \item \textbf{\modLcd} LCD-display för att visa parametrar under körning i
        felsökningssyfte.
    \item \textbf{IQEXO3} Kristalloscillator som systemklocka till AVR.
    \item \textbf{Tryckknapp} Knapp som används till RESET.
\end{itemize}

\subsubsection{Pinnar}
Nedan finns de pinnar som används på mikrokontrollern.
\begin{itemize}
   \item \textbf{PA4-PA7} Datasignaler till LCD (DB4-DB7).
   \item \textbf{PB0} RS-signal till LCD.
   \item \textbf{PB1} Enable-signal till LCD.
   \item \textbf{PB4-PB7} SS, MOSI, MISO och SCLK för SPI-bussen.
   \item \textbf{PC2-PC5} TCK, TMS, TDO, TDI för JTAG.
   \item \textbf{PD4} PWM-signal till drivmotor.
   \item \textbf{PD5} PWM-signal till svängservo.
   \item \textbf{RESET} Resetsignal.
   \item \textbf{XTAL1} Klocka från kristalloscillatorn.
   \item \textbf{AREF} A/D omvandlare används ej, men referensspänning till A/D omvandlare.
\end{itemize}
R/W signalen till LCD är jordad då vi ej behöver läsa från LCD. Anod
och katod används ej eftersom vi ej behöver någon bakgrundsbelysning. Inga
fler signaler behövs så pinnarna på mikrokontrollen är tillräckliga.

Mikrokontrollen behöver endast kommunicera via SPI, utföra enkel reglering och
därefter skapa en PWM-signal för motorn och svängservo. Inget av detta är
särskilt krävande och klaras utan problem av ATMega1284. Kontrollern har
ett flash-minne på 128kB, vilket är mer än tillräckligt för att lagra
programmet.

\subsection{Mjukvaruimplementation} Styrmodulens program  är skrivet i C och
består av ett avbrott som tar emot ett kommando och eventuell data från
kommunikationsmodulen. Kommandona bestämmer i programmets main-loop vad som ska
hända med den mottagna datan. Det kan vara att t.ex. sätta hastigheten eller
att uppdatera regleringskonstanter. 

\subsubsection{Huvud-loop och reglering} I huvudloopen initialiseras först både
SPI och PWM. Sedan används det mottagna kommandot för att bestämma vad den
mottagna datan ska användas till. Om kommandot säger att datan innehåller
direkta hastighets- eller rotatationsvärden och dessa värden är giltiga så
sätts dessa direkt till det nuvarande PWM-värdet. Den typen av kommando används
när bilen opererar i manuellt läge. Om det är felvärden som tas emot används
dessa för att reglera bilens hastighet och styrning och därefter skicka
motsvarande PWM-signaler till motorn och servon.  Regleringen fungerar genom
att räkna ut hastigheten eller radien $v[n]$ med hjälp av ett felvärde $e[t]$
så att \begin{equation*} v[n] = k_p \cdot e[n] + k_d \cdot (e[n]-e[n-1])
\end{equation*} där $k_p$ och $k_d$ är konstanter som kan tas emot från
kommunikationsmodulen via andra kommandon som är specifika för att skicka
nya K-värden. Den första termen står för att justera utefter det nuvarande
felet. Den andra termen står för att justera utefter den nuvarande
förändringen av felet. I sektion \ref{sec:wlproto} finns en detaljerad
lista på alla de kommandon som kan förekomma och hur de används.

\subsubsection{Initialisering} Det första som sker i main programmet är att
portar och register på microkontollen initialiseras till rätt användningsläge.

PWM-signalerna använder sig utav den inbyggda timern TIMER1 och den
initialeseras till att arbeta i fas- och frekvenskorrekt PWM vilket fungerar
bra för servomotorer. Timern ger sedan ut PWM-pulser på portarna PD4 och PD5
och därför sätts också dessa portar som utsignaler i datariktningsregistret
DDRD. 

SPI för styrmodulen initialiseras och aktiveras genom att sätta
SPI-interrupt-enable och SPI-enable i SPCR registeret på microcontrollen.

\subsubsection{Avbrott} \label{sec:ctrl-int}
När SPI-kontrollern har tagit emot en byte av data sätter den SPIF i SPSR (SPI
Status Register) och aktiverar ett avbrott. Avbrottsrutinen kan kontrollera
flaggan för att avgöra att kommunikationsmodulen vill skicka felvärden och
börja ta emot värden.

\end{document}
