\documentclass[tekniskrapport/tech.tex]{subfiles}
 
\begin{document}
\section{Slutsatser}
Under projektets gång har många delar gått bra men det finns flera saker som
hade kunnat förbättras för att få produkten att kännas mer proffessionell och
leveransfärdig . 

Bilen hade kunnat köra om hinder på vägen för att
säkerhetsställa en snabbare framkomst på banan. Detta har räknats som en
prioritetspunkt 2 vilket är en anledning till funktionalitetens
bortprioritering.

Att visa bilden på det grafiska användargränssnittet har försummats då det
skulle ta mycket tid att implementera på ett bra sätt. Multitrådning hade
behövts då en video-stream på samma tråd som bildbehandlingen hade kunnat
resultera i en lägre bildfrekvens för bildbehandlingen vilket inte är önskvärt
om bilen ska kunna hålla en bra hastighet.

Kartan hade visuellt kunnat representeras på ett mer verkligt sätt genom att
införa böjda kanter istället för enbart raka linjer mellan noder. Det hade
varit mer överskådeligt men skulle krävas mer jobb från gruppens sida. Det
skulle även underlättat om det var möjligt att flytta utsatta noder. Det är
något som skulle vara ganska lätt att implementera men har inte prioriterats
eftersom den nuvarande kartan fungerar bra som den är.

Virkortet där alla komponenter sitter är full med spretande sladdar som hade
kunnat undvikas genom att beställa ett färdiggjort kort designat efter
systemets kretsschema. Det hade gjort att taxin utseendemässigt sett mer
proffessionell ut men skulle också eliminera eventuella glapp i virkablarna.

Bussens kommandon hade kunnat designats från grunden istället för som det ser
ut i dagsläget. Innehållet som måste kunna skickas har ändrat fram och tillbaka
under projektets gång och kommandona likaså. En omdesign av kommandona skulle
innebära snyggare och effektivare kod.

Främsta anledningen till att majoriteten av ovanstående funktioner inte finns
med i slutprodukten är tidsbrist. Den tid som var tilldelad projektgruppen
räckte helt enkelt inte till för att implementera allt som önskades. Tiden
användes främst till att täcka alla de grundkrav som finns listade i
kravspecifikationen. Med mer tid till gruppens förfogande hade den färdiga
produkten innehållit fler funktioner och varit snyggare och stabilare.
\end{document}

