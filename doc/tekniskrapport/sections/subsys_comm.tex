\documentclass[tekniskrapport/tech.tex]{subfiles}

\begin{document}

\section{Kommunikationsmodul}
Kommunikationsmodulen styr bilen och kommunicerar med fjärrklienten.
Sensorvärden från sensormodulen tolkas av kommunikationsmodulen som därefter
skickar felvärden till styrmodulen.

Bildbehandling utförs av kommunikationsmodulen för att
avgöra bilens position i vägfilen och upptäcka stopplinjer. Med hjälp av
kamerans bilder på vägen skapas ett felvärde som kan användas för att justera
taxins riktning.

Kommunikation med fjärrklienten sköts av kommunikationsmodulen för
att skicka sensorvärden och annan relevant information. Modulen tar
även emot uppdrag.

Autonomitet utförs av kommunikationsmodulen för att utföra uppdraget.
Kommunikationsmodulen hämtar sensordata från sensormodulen och bildbehandlar
kamerabilderna för att utföra beslut i realtid. Besluten skickas i form av fel-
och styrvärden till styrmodulen.

Kommunikationsmodulens program är skriven i både C och C++. Bildbehandlingen är
skriven i C++ medan resten av programmet är implementerad med C. De olika
delarna kompileras separat och länkas sedan ihop med kompilatorn för C++.
Programmet består av tre trådar; en huvudtråd som hanterar uppdrag och
bildbehandlar, samt en tråd som sköter kommunikation via SPI och en som svarar
på kommandon från fjärrklienten via TCP/IP. SPI:n och servern sköts i separata
trådar eftersom de väntar mycket och blockerar den nuvarande tråden.

\subsection{Hårdvaruimplementation} 
Kommunikationsmodulen har implementerats i en Raspberry Pi 3B. Raspberry Pi 3B,
har en fyrkärnig processor med en hastighet på 1.2 GHz vilket har kunnat
utnyttjas inte minst vid bildbehandlingen. Med 40 pinnar som kan utnyttjas som
spänningskällor, jord, programmerbara GPIO-portar samt seriella bussar, har
detta utnyttjats för att med SPI-protokollet använda Raspberry Pi:n som
\emph{master} för två slavar. Då kommunikationen med en fjärrklient sker via
WiFi har det inbyggda 802.11n trådlösa LAN:et varit fullt kapabelt. Det
trådlösa nätverkskortet hade även kunnat strömma video. Mitt på Pi:en hittar
man en CSI-port som används för att ansluta Raspberry Pi-kompatibla kameror.
Kameran som används på kommunikationsmodulen är en Raspberry Pi-kamera med
påsatt vidvinkelobjektiv som kan se med en vy på 160 grader.

\subsection{Programstruktur}
Programmet på kommandomodulen består av följande filer

\begin{labeling}{wwwwwwwwww}
    \item[server.c] Server som anropar kommandofunktioner vid förfrågan från
        klient från separat tråd.
	\item[spi.c] Lågnivåfunktioner för SPI.
    \item[bus.c] Hantering av SPI-buss, utför kommandon på bussen och anropar
        signalhanterare från separat tråd.
    \item[ip/img\_proc.cpp] Bildbehandling med OpenCV.
    \item[objective.c] Utförarande av uppdrag.
    \item[main.c] Huvudloop och implementation av signalhanterare för buss och
        server.
\end{labeling}

\paragraph{Huvudtråden} startar servern och busshanteraren. Därefter startar
den huvudloopen som antingen tar styrvärden utifrån användarens kommandon via
servern eller från uppdragshanteraren med hjälp av bildbehandling av den
senaste bilden från kameran. Därefter schemaläggs skrivning av de nya
styrvärdena till styrmodulen. Dessutom schemaläggs hämtning av sensorvärden
från sensorerna.

\paragraph{Servern} erhåller en sockel för nya anslutningar och en för att
skicka och ta emot meddelanden för en aktiv anslutning. Servern kan därmed
endast erhålla en anslutning i taget. Anslutningssockeln är bunden till en port
mellan 9000 och 9100. När en anslutning förfrågas eller om ett meddelande tas
emot väcks servern och accepterar eller tar emot meddelande. Om en
anslutningsförfrågan tas emot skapas en ny sockel för meddelanden och
överskrider den tidigare sockeln om den finns. Om ett meddelande med ett
giltigt kommando tas emot anropas kommandots funktion som utför sin handling
och genererar ett svarsmeddelande som servern sedan vidarebefogar till
klienten. Därefter sover servern igen tills en ny förfrågan tas emot.

\paragraph{Busshanteraren} fungerar på liknande sätt som servern. Den sover
tills en förfrågan sker. Förfrågningarna kommer däremot från andra trådar i
programmet istället för klienten. Om flera förfrågningar sker innan servern har
utfört den nuvarande kommer de nya meddelandena köas i sockeln av
operativsystemet. Det finns dock ingen sockel att köa förfrågningar i för
busshanteraren. Därför har en länkad lista implementerats för att kunna erhålla
flera förfrågningar. När en förfrågan sker läggs den sist i kön och
busshanteraren väcks om den sover. Om kön redan innehåller en förfrågan med
samma kommandotyp ersätts den med den nya förfrågan. Busshanteraren utför
därefter förfrågan via SPI och om kommandot gick igenom anropas
signalhanteraren för det kommandot. Om kommandot inte gick igenom läggs
kommandot sist i kön om ett nyare förfrågan av samma typ inte redan finns i
kön. Annars kastas den äldre förfrågan bort.

\subsection{Bildbehandling}
Bildbehandlings processen sker genom ett program skrivet i C++. För att utföra
alla nödvändiga rutiner inom bildbehandling, vilka möjliggör vägens linjer
upptäckande, används programvarubiblioteket OpenCV.

\subsubsection{Upptäckande av linjer}

\begin{wrapfigure}{r}{0.26\linewidth}
    \vspace*{-8mm}
    Originalbild:\\[1mm]
    \includegraphics[width=\linewidth]{\figures/original.jpg}
    \label{fig:gauss}
    Gauss:\\[1mm]
    \includegraphics[width=\linewidth]{\figures/gauss.png}
    \label{fig:gauss}
    Gråskala:\\[1mm]
    \includegraphics[width=\linewidth]{\figures/monochromatic.png}
    \label{fig:monochromatic}
    Threshold:\\[1mm]
    \includegraphics[width=\linewidth]{\figures/threshhold.png}
    \label{fig:threshhold}
    Canny edges:\\[1mm]
    \includegraphics[width=\linewidth]{\figures/edge_detection.png}
    \label{fig:edge_det}
    Maskning:\\[1mm]
    \includegraphics[width=\linewidth]{\figures/mask_roi.png}
    \label{fig:mask}
    Hough transform:\\[1mm]
    \includegraphics[width=\linewidth]{\figures/lines.png}
    \label{fig:lines}
\end{wrapfigure}

Upptäckandet av linjer/kanter på banan sker genom hanteringen av varje bild
programmet tar emot från kameran. För varje bild gås en process igenom som
börjar med att filtrera bilden för att först underlätta kanternas upptäckande
och sedan linjerna inom ett bestämt område av intresse (Region of Interest
ROI).

Processen börjar med att filtrera bilden med det så kallade Gauss-filtret där
bilden faltas med en Gauss-kernel med syfte att filtrera bort brus. Den
filtrerade bilden omvandlas sedan till en gråskalabild vilket underlättar
bildbehandlingen då mindre information behöver tas hänsyn till.

Nästa steg är att segmentera bilden genom att använda
\emph{threshold}-funktionen som möjliggör att skilja mellan olika objekt som
önskas analyseras baserad på hur intensiteten hos varje pixel varierar mellan
objektet och bakgrunden. Programmet använder sig av Otsu-metoden vilken själv
räknar ut ett tröskelvärde som bäst passar bilden. Utifrån tröskelvärdet blir
varje pixel antingen svart eller vit så att en binärfärgad bild skapas.

När bilden är binär används Canny-operationen vilken identifierar alla kanter
som finns i bilden.

Som tidigare nämnts, upptäcks linjerna inom ROI. Eftersom kameran ser långt
fram och även delar som är utanför banan är det bara intressant att se den
närmaste delen av vägen. Därför väljs en ROI som bara innehåller väsentliga
delar så att tolkningen endast behöver ta hänsyn till vänsterkant, högerkant
och stopplinje. ROI är en polygon av 6 punkter som maskar resultatet från
Canny-operationen.

Efter att bilden är filtrerad och ROI maskad, används Hough-transformen för att
hitta linjer enligt bestämda parametrar som varje linje ska uppfylla i
transformen för att bli klassad som linje. Programmet använder sig av en
probabilistisk metod vilket ökar beräknings hastighet utan att minska
precisionen. Efter användningen av Hough-transformen fås en lista av linjer som
programmet sedan kan tolka.

\subsubsection{Tolkning av linjer}
Det första som görs med alla linjer som hittas på bilden är att klassificera de
som vänsterlinje, högerlinje, stopplinje eller annan linje. Dessa färgas gröna,
blå, röda och gråa respektive. Därefter beräknas x-positionen av filens
högerkant vid ett $y=l_y$ genom att räkna ut medianvärdet av x-positionerna där
alla högerlinjer korsar $y=l_y$. På samma sätt räknas vänsterkantens position i
x-led ut. Därifrån kan filens bredd $w$ och mittpunkten $l_x$ räknas ut vid
$y=l_y$. Felvärdet beräknas därefter genom att jämföra $l_x$ med bildens
mittpunkten.

\begin{wrapfigure}{r}{0.3\linewidth}
    \includegraphics[width=\linewidth]{tekniskrapport/figures/opencv.jpg}
    \caption{Klassificering och tolkning av linjer från kamera.}
\end{wrapfigure}

Vänster och högerlinjerna används dessutom för att uppskatta filens riktning.
Detta görs genom att addera alla sidolinjer som vektorer och därmed få en
vinkelsumma $\vec{v}$ (gul färg) som har en genomsnittlig riktning. En linje
$D$ som korsar $\vec{l}=(l_x, l_y)$ och har riktningsvektorn $\vec{v}$ används
för att klassificera nästa bilds linjer. Linjer som korsar $D$ och har en stor
vinkel till $D$ klassificeras som stopplinjer. Linjer vars båda punkter
befinner sig till höger om $D$ och inte har en särskilt stor vinkel till $D$
klassificeras som en högerlinje. På samma sätt klassificeras vänsterlinjer.

Om antingen vänster eller höger kant inte skulle upptäckas används den
uppskattade bredden för att räkna ut var mittpunkten av filen är.

Linjer som klassificeras som stopplinjer används för att bestämma om en
stopplinje har korsats. Stopplinjens position i y-led uppskattas utifrån den
tidigare bilden. Endast linjer som är inom ett visst avstånd till den
uppskattade positition antas vara del av stopplinjen. Till en början antas den
vara högst upp i ROI:n. När stopplinjen har upptäckts hela vägen från toppen av
ROI:n till botten av bilden utan att gå för långt ifrån uppskattningen
registreras en passering av stopplinjen. Om linjen inte skulle upptäckas under
en bild flyttas uppskattningen tillbaka till toppen av ROI:n.
  
\subsection{Uppdrag} \label{sec:comm-mission}
Uppdragshanteraren körs i huvudloopen. Det första uppdragshanteraren gör för
varje iteration är att kolla om ett nuvarande kommando är aktivt. Annars hämtar
den ett nytt från kommandokön. Om kön är tom är uppdraget avslutat. Om kön
istället har ett nuvarande kommando utförs bildbehandling och styrvärdena sätts
till normalhastighet samt styrfelvärde från bildbehandlingen. Därefter anropas
kommandofunktionen som kan skriva över dessa samt ändra inställningar för
bildbehandlingen. Funktioner returnerar om kommandot är avslutat eller inte. Om
kommandot är färdigt tas kommandot bort från nuvarande och ett nytt hämtas vid
nästa iteration.

\clearpage
Inom kommandon håller en variabel \mono{pos} koll på var i kommandot bilen
befinner sig i. Vid varje nytt kommando sätts den till \mono{BEFORE\_STOP}
vilket innebär att bilen inte har passerat kommandots stopplinje än. När
bildbehandlingen registrerar att en stopplinje har passerats sätts \mono{pos}
till \mono{AFTER\_STOP}. Kommandon kan även ändra \mono{pos} själva för att
kunna ha flera tillstånd. Vid varje ändring av \mono{pos} registreras den
nuvarande tiden och distansen. Nedan följer en lista av alla tillgängliga
kommandon och hur de utförs.

\begin{labeling}{wwww}
    \item[\commIgnore] Avslutar så fort \mono{pos} är \mono{AFTER\_STOP}.
    \item[\commStop] Sätter hastigheten till noll när \mono{pos} är
        \mono{BEFORE\_STOP} och stopplinjen är synlig.  Avslutar när \mono{pos}
        är \mono{AFTER\_STOP} om kommandot är sist i kön, annars väntar fem
        sekunder och avslutar därefter.
    \item[\commPark] Saktar ner och svänger höger när \mono{pos} är
        \mono{AFTER\_STOP}. Låter bildbehandlingen därefter ta över i en viss
        distans. Därefter sätter \mono{pos} till \mono{PARKED} och stannar. Om
        kommandot var sist i kön avslutas det där. Annars sätter den \mono{pos}
        till \mono{UNPARKING} efter fem sekunder och svänger vänster i en viss
        distans och därefter avslutar.
    \item[\commEnter] Låter bildbehandlingen ignorera vänsterlinjen när
        \mono{pos} är \mono{AFTER\_STOP} så att bilen följer högerlinjen. Efter
        en viss distans avslutar kommandot.
    \item[\commContinue] Låter bildbehandlingen ignorera högerlinjen när
        \mono{pos} är \mono{AFTER\_STOP} så att bilen följen vänsterlinjen och
        fortsätter i rondellen.
    \item[\commExit] Låter bildbehandlingen ignorera vänsterlinjen när
        \mono{pos} är \mono{AFTER\_STOP} så att bilen följer högerlinjen och
        åker ut ur rondellen. Avslutar kommandot efter en viss distans.
\end{labeling}

\end{document}
