\documentclass[tekniskrapport/tech.tex]{subfiles}

\newcommand{\wlcomm}[3]{\mono{#1} & #2 & #3 \\}

\makeatletter
\define@key{bytefield}{bfopts}[]{\setkeys{bytefield}{bitwidth=2.8mm}}
\makeatother

\newcommand*{\vcenteredhbox}[1]{\begingroup
\setbox0=\hbox{#1}\parbox{\wd0}{\box0}\endgroup}

\begin{document}

\section{Kommunikation}
Kommunikation sker mellan fjärrklienten och kommunikationsmodulen samt mellan
kommunikationsmodulen och vardera mikrokontroller i styr- och sensormodulen.
Denna sektion specificerar hur kommunikationen uppnås samt hur informationen
som växlas uttrycks av sändaren och tolkas av mottagaren.

\subsection{Trådlös länk}
Taxin och fjärrklienten kommunicerar via TCP/IP. Kommunikationsmodulen och
fjärrklienten ansluter till ett gemensamt WLAN. Därefter binder
kommunikationsmodulen sin IP-address och en port till en sockel som
fjärrklienten ansluter till. Endast en anslutning erhålls av servern, om flera
klienter ansluter erhålls anslutningen till den senast anslutna.

\subsubsection{Protokoll}
\label{sec:wlproto}
Kommunikationsmodulen väntar på inkommande instruktioner via sockeln och
exekverar dem när de tas emot. Instruktioner skickas i form av strängar och kan
inkludera argument. För instruktioner utan argument skickas endast en sträng
med instruktionens namn; till exempel \mono{get\_sensor}. För instruktioner med
argument består strängen av instruktionens namn följt av ett kolon och ett
eller flera argument. Flera argument separeras med kommatecken; till exempel
\mono{set\_miss:\commIgnore,\commIgnore,\commPark}.

{\setlength{\tabcolsep}{12pt}
\renewcommand{\arraystretch}{1.6}
\begin{longtable}{p{3.5cm}p{2.8cm}p{7cm}}
    \bfseries Instruktion & \bfseries Argument & \bfseries Handling \\\hline
    \wlcomm{help}{inga}{Returnerar alla tillgängliga serverkommandon,
    separarerade med mellanslag.}
    \wlcomm{get\_sensor}{inga}{Returnerar mätvärden från sensorer samt
    felvärden och motorstyrvärden, separerade med mellanslag. Värdenas
    betydelse och ordning är följande: distans på frontsensor, distans på
    sidosensor, hastighet, total körd distans, felvärde i körfil,
    motorhastighet, motorrotation. Alla värden är i SI-enheter förutom
    felvärdet och motorrotationen. De är istället tal mellan -1 och 1, där -1
    motsvarar längst till vänster och 1 motsvarar längst till höger.}
    \wlcomm{set\_miss}{kommandonamn}{Sätt en kö av kommandon som ska utföras
    vid stopplinjer. Kommandona är specificerade i sektion
    \ref{sec:comm-mission}. Om ett tidigare uppdrag finns i kön skrivs det
    över.}
    \wlcomm{app\_miss}{kommandonamn}{Samma funktion som \mono{set\_miss} men
    istället för att avbryta och skriva över det nuvarande uppdraget, läggs
    listan av uppdraget till efter det nuvarande uppdraget.}
    \wlcomm{get\_miss}{inga}{Returnerar alla kommandon i det nuvarande
    uppdraget. KommandoN är separarade med mellanslag. Om den nuvarande kön av
    kommandon är tom, kommer strängen som returneras att vara tom.}
    \wlcomm{set\_auto}{bool}{Välj mellan autonom eller manuell körning.
    \mono{T} sätter autonom körning och ett \mono{F} sätter manuell körning.}
    \wlcomm{set\_vel}{flyttal}{Sätt motorstyrvärde för hastigheten, ignoreras
    om uppdrag är aktivt.}
    \wlcomm{set\_rot}{flyttal}{Sätt motorstyrvärde för svängradien, ignoreras
    om uppdrag är aktivt.}
    \wlcomm{set\_vel\_kp}{flyttal}{Sätt proportionerlig konstant för
    regleringen av hastigheten}
    \wlcomm{set\_vel\_kd}{flyttal}{Sätt derivatakonstant för regleringen av
    hastigheten}
    \wlcomm{set\_rot\_kp}{flyttal}{Sätt proportionerlig konstant för
    regleringen av svängradien}
    \wlcomm{set\_rot\_kd}{flyttal}{Sätt derivatakonstant för regleringen av
    svängradien}
    \wlcomm{shutdown}{inga}{Stanna taxin och stäng av kommunikationsmodulen.}
\end{longtable}
}

\subsection{Mellan taxins moduler}
Mellan sensormodulen, kommunikationsmodulen och styrmodulen används
gränssnittet SPI. Kommunikationsmodulen agerar som \emph{master} medan
sensormodulen och styrmodulen agerar som \emph{slaves}. Varje modul är kopplad
till en gemensam SCLK-signal för synkronisering, en MISO-signal för data till
master:n och MOSI för data från master:n som visat i figur \ref{fig:bus_inter}.
Sensormodulen skickar sensorvärden till kommunikationsmodulen och
kommunikationsmodulen skickar styrkommandon till styrmodulen.
\begin{figure}[h]
    \centering
    \subfile{\figures/bus_spi}
    \caption{Den gemensamma databussen mellan modulerna.}
    \label{fig:bus_inter}
\end{figure}

\subsubsection{Kommandostruktur}
Kommunikationsmodulen agerar som \emph{master} och har därmed full kontroll
över bussen. Kommunikationsmodulen skickar därför kommandon till slavarna som
antingen är läs eller skrivkommandon. 

Ett skrivkommando från kommunikationsmodulen är strukturerad som i figur
\ref{bf:spi-write}. Under skrivkommandon skickar kommunikationsmodulen data
till en slav. De fyra första högre bitarna av det första ordet från
kommunikationsmodulen berättar vilket kommando som skickas. Då vet slaven om
det är ett läs- eller skrivkommando och hur mycket data som ska skickas eller
ta emot. I fallet för skrivkommando består de fyra lägre bitarna av det första
ordet av en kontrollsumma för data som följer. Därefter följer $n$ antal
dataord som slaven tar emot. Om kontrollsumman matchar den mottagna datan
kommer slaven svara med en ACK förspecificerad för slaven.
\begin{figure}[h]
    MOSI
    \begin{bytefield}[bfopts]{40}
        \vcenteredhbox{
            \bitbox{4}{CMD}\bitbox{4}{SUM}
            \bitbox{8}{DATA 0}
            \bitbox[lr]{8}{$\cdots$}
            \bitbox{8}{DATA $n$}
            \emptybox{8}
        }
    \end{bytefield}\\[1mm]
    MISO
    \begin{bytefield}[bfopts]{40}
        \vcenteredhbox{
            \emptybox{8}
            \emptybox{8}
            \bitbox[lr]{8}{$\cdots$}
            \emptybox{8}
            \bitbox{8}{ACK}
        }
    \end{bytefield}
    \caption{Bitsekvensen för ett skrivkommando.}
    \label{bf:spi-write}
\end{figure}

Ett läskommando är strukturerat som i figur \ref{bf:spi-read}.
Kommunikationsmodulen specificerar först kommandot, därefter svarar slaven med
en ACK samt en kontrollsumma för datan som följer.
\begin{figure}[h]
    MOSI
    \begin{bytefield}[bfopts]{24}
        \vcenteredhbox{
            \bitbox{4}{CMD}\emptybox{4}
            \emptybox{8}
            \emptybox{8}
            \emptybox{8}
            \bitbox[lr]{8}{$\cdots$}
            \emptybox{8}
        }
    \end{bytefield}\\[1mm]
    MISO
    \begin{bytefield}[bfopts]{24}
        \vcenteredhbox{
            \emptybox{8}
            \bitbox{8}{ACK}
            \emptybox{4}\bitbox{4}{SUM}
            \bitbox{8}{DATA 0}
            \bitbox[lr]{8}{$\cdots$}
            \bitbox{8}{DATA $n$}
        }
    \end{bytefield}
    \caption{Bitsekvensen för ett läskommando.}
    \label{bf:spi-read}
\end{figure}

\subsubsection{Kontrollsumma och ACK}
Kontrollsumman är implementerad för att undvika att felaktig data skickas eller
tas emot. Kontrollsumman beräknas genom att utföra XOR för de fyra högre och
lägre bitarna för varje ord i datan samt de fyra bitarna för kommandot.

ACK:en är implementerad för att låta kommunikationsmodulen veta att slaven
lyssnar och att datan har kommit fram intakt. Sensormodulens ACK är
$\mono{0x53}$ och styrmodulens är $\mono{0xc7}$.

\subsubsection{Kommandon}
En lista av alla implementerade kommandon finns i bilaga \ref{bcmds}.
Sensormodulens kommando \mono{GET} används för att hämta sensordata från
sensormodulen. Under autonom körning används därefter \mono{VVL} för att sätta
bilens hastighet och \mono{RER} för att reglera styrningen.

\end{document}
