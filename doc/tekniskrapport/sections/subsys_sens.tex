\documentclass[tekniskrapport/tech.tex]{subfiles}

\begin{document}

\section{Sensormodul}
Hos sensormodulen hanteras de olika värdena från bilens sensorer och
kommunicerar med kommunikationsmodulen vid efterfrågan av värden via SPI.

\subsection{Funktion} Sensormodulen hämtar filtrerade värden från sensorerna,
värden som först hanteras hos mikrokontrollens AD/omvandlare. Sensormodulen
har i uppgift att läsa in värdena, uppdatera dem kontinuerligt, omvandla dem
till linjära enheter (meter eller meter/sek) och skicka dem vidare till
kommunikationsmodulen.

\paragraph{Filtrering} Eftersom sensorerna kan bli påverkade av eventuella
störningar, används ett filter som filtrerar bort störningar. Ett filter
används vid varje koppling sensor-mikrokontroller.

\paragraph{Sensorer} Bilen innehåller två avståndsmätaren placerade på bilens
fram och bakre högersida. Sensorn på framsidan används för att upptäcka hinder
medan den andra sensor används för att detektera hinder vid omkörning. Dessutom
innehåller bilen halleffektsensorer vid båda bakhjulen, sensorer som används
för att mäta bilens hastighet samt avståndet från startpunkten.

\subsection{Hårdvaruimplementation} Sensormodulen består av en mikrokontroller,
och fyra sensorer; en avståndsmätare på framsidan, en avståndsmätare på bakre
högersida samt två halleffektsensorer placerade på båda bakhjulen.
Mikrokontrollern är en ATmega1284 som är klockad av en kristalloscillator på 16
MHz. Mikrokontrollern är kopplad
till en LCD-display på vilken de olika värden från sensorerna visas; avstånd
till hinder från både fram och höger sensor samt avståndet som bilen har från
startpunkten. Ett detaljerat kretsschema över modulen finns i bilaga
\ref{cdiag:sens}.

Avståndsmätarna skickar kontinuerligt en spänning (den analoga signalen) till
mikrokontrollens AD-omvandlare. Den digitala signalen hanteras hos
sensormodulen så att den onvandlas till meter för att bli skickad vidare till
kommunikationsmodulen som ett avstånd till hinder. Avständmätaren placerad på
bilens framsida är av typen GP2Y0A02YK vars analog utsignal är mellan 0V och
3V, medan sensorn placerad på bilens bakre högersida är av typen GP2Y0A41SK
vars utsinal är en analog spänning mellan 0V and 3.2V. Hos båda typer av sensor
gäller att ju närmare hindret befinner sig desto högre spänningen går. 

Som tidigare nämnts, används det ett brusfilter vid varje
sensor-mikrokontroller koppling. Varje filter består av en 18K resistans och en
100nF jordad kondesator; ett passivt RC-filter.

\subsubsection{Budget}
Nedan är externa produkter som har använts vid sensormodulens konstruktion.
\begin{itemize}
	\item \textbf{\modMicrocontroller} ATMega1284, AVR. 
    \item \textbf{\modDistf} Optisk avståndsmätare GP2Y0A02YK (20-150cm).
    \item \textbf{\modDists} Optisk avståndsmätare GP2Y0A41SK (4-30cm).
    \item \textbf{\modLcd} LCD-display.
    \item \textbf{IQEXO3} Kristalloscillator som systemklocka till AVR.
\end{itemize}
Modulen använder även en {\modJtag} och tryckknapp (reset knapp) som delas med
styrmodulen.

\subsubsection{Kontroll och bedömning}
Nedan är följande pinnar som används på mikrokontrollern.
\begin{itemize}
   \item \textbf{PA0 \& PA1} Vardera port får insignal från en avståndsmätare.
   \item \textbf{PB4-PB7} SS, MOSI, MISO och SCLK för SPI-bussen.
   \item \textbf{PC2-PC5} TCK, TMS, TDO, TDI för JTAG.
   \item \textbf{PA2} LCD-enable.
   \item \textbf{PA3} Register-select signal till LCD-display.
   \item \textbf{PD2} Avbrottssignal från en odometer.
   \item \textbf{PD3} Avbrottssignal från den andra odometern.
   \item \textbf{PD4-PD7} Databuss till LCD-display.
   \item \textbf{RESET(9)} Knapp till reset för MCU.
   \item \textbf{XTAL1} Klocka från Kristalloscillatorn. 
   \item \textbf{AREF} Referensspänning för AD-omvandlare på 3.3V.

\end{itemize}
Mikrokontrollen kommunicerar med kommunikationsmodulen via SPI och hanterar de
värdena från alla sensorer. Från avståndsmätarna beräknar kontrollen avståndet
till hinder genom att omvandla den digitala signalen till meter som hämtas från
AD/omvandlare. Från halleffektsensorerna beräknar kontrollen bilens hastighet
samt bilens avstånd från startpunkten genom att använda sig av antalet avbrott
som sensorn orsakar samt hjulens omkrets.

\subsection{Mjukvaruimplementation} 
Mjukvaran på sensormodulen är skriven i programmeringspråk C, en fil där alla
sensormodulens uppgifter hanteras. I filen implementeras en oändlig while loop
där värdena från alla sensorer uppdateras kontinuerligt medan modulen väntar på
avbrottet som orsakas via SPI när kommunikationsmodulen efterfrågar värden. Vid
SPI avbrott, hämtas de senaste sensorvärdena och skickas vidare till
kommunikationmodulen. Avbrottet aktiveras som beskrivet i sektion
\ref{sec:ctrl-int}

Processen för att få avståndet till hinder (i meter) från varje avståndmätare
sker genom att hämta den digitala signalen (hos AD-omvandlaren) och omvandla
värdet till meter. Den formlen som används vid omvandligen från adc värdet till
meter beror på sensors typ och är resultatet av den linjära regressionen
applicerade med olika mätningar.

Varje bakhjul innehåller tio magneter vilka vardera orsakar ett avbrott. I
programmet hanteras dessa avbrott, som orsakas från båda hjul, så att de tillsammans
med bakhjulets omkrets används för att räkna ut bilens hastighet samt avståndet
bilen har åkt sedan startpunkten.

\end{document}
