\documentclass[tekniskrapport/tech.tex]{subfiles}

\begin{document}

\section{Sensormodul}
Hos sensormodulen hanteras de olika värden från bilens sensorer (avståndmätare och halleffektsensorerna) vilka sedan skickas som mätdata till kommunikationsmodulen.

\subsection{Funktion}
Sensormodulen hämtar filtrerade värden från sensorerna, omvandla
till linjära enheter och skicka vidare till kommunikationsmodulen.

\paragraph{Sensorer} Bilen innehåller två avståndmätare placerade på bilens fram och bakre högersida. Sensorn på framsidan används för att upptäcka hinder medan den andra sensor används för att detektera hinder vid omkörning. Dessutom innehåller bilen halleffeksensorer vid båda bakhjulen vilka sensormodulen används för att mäta bilens hastighet samt avståndet från startpunkten. 

\paragraph{Filtrering} Eftersom sensorerna kan bli påverkade av eventuella
störningar används, vid varje sensor, ett filter som filtrerar bort
störningar.

\subsection{Hårdvaruimplementation} Sensormodulen består av en
mikrokontroller, och fyra sensorer; en avståndsmätare på framsidan, en
avståndsmätare på bakre högersida samt två halleffektsensorer placerade på båda bakhjulen. Mikrokontrollern är klockad av en kristalloscillator på 16 MHz och även kopplad till en LCD-display på vilken de olika värden från sensorerna visas; avstånd till hinder från både fram och höger sensor samt avstånd bilen har från startpunkten. Ett detaljerat kretsschema över modulen finns i bilaga
\ref{cdiag:sens}??.

Avståndsmätarna skickar kontinuerligt en spänning (den analoga signalen) till mikrokontrollens AD-omvandlare. Den digitala signalen hanteras hos sensormodulen så att den onvandlas till meters enheten för att bli skickad vidare till kommunikationsmodulen som ett avstånd till hinder.

Som tidigare nämnts, används det ett brusfilter vid varje sensor koppling. Filter som består av en 18K resistans och en 100nF jordad kondesator, ett passivt RC-filter.

\subsubsection{Budget}
Nedan är externa produkter som har använts vid sensormodulens konstruktion.
\begin{itemize}
	\item \textbf{\modMicrocontroller} ATMega1284, AVR. 
    \item \textbf{\modDistf} Optisk avståndsmätare GP2Y0A02YK (20-150cm). Placerad på bilens framsidan.
    \item \textbf{\modDists} Optisk avståndsmätare GP2Y0A41SK (4-30cm). Placerad på bilens bakre högersidan.
    \item \textbf{\modLcd} LCD-display för att visa parametrar under körning
    i felsökningssyfte.
    \item \textbf{IQEXO3} Kristalloscillator som systemklocka till AVR.
\end{itemize}
Modulen använder även en {\modJtag} och tryckknapp (reset knapp) som delas med styrmodulen.

\subsubsection{Kontroll och bedömning}
Nedan är följande pinnar som används på mikrokontrollern.
\begin{itemize}
   \item \textbf{PA0 \& PA1} Vardera port får insignal från en avståndsmätare.
   \item \textbf{PB4-PB7} SS, MOSI, MISO och SCLK för SPI-bussen.
   \item \textbf{PC2-PC5} TCK, TMS, TDO, TDI för JTAG.
   \item \textbf{PA2} LCD-enable.
   \item \textbf{PA3} Register-select signal till LCD-display.
   \item \textbf{PD2} Avbrottssignal från en odometer.
   \item \textbf{PD3} Avbrottssignal från den andra odometern.
   \item \textbf{PD4-PD7} Databuss till LCD-display.
   \item \textbf{RESET(9)} Knapp till reset för MCU.
   \item \textbf{XTAL1} Klocka från Kristalloscillatorn. 
   \item \textbf{AREF} Referensspänning för AD-omvandlare på 3.3V.

\end{itemize}
Mikrokontrollen kommunicerar med kommunikationsmodulen via SPI och hanterar de värdena från alla sensorer. Från avståndmätare beräknar kontrollen avståndet till hinder genom att omvandla till meter den digitala signalen, hämtad från AD/onvandlare. Från halleffektsensorerna beräknar kontrollen bilens hastighet samt bilens avstånd från startpunkten genom att använda sig av antalet avbrott som sensorn orsakar samt hjulens omkrets. 


\subsection{Mjukvaruimplementation} 
Hos sensormodulen används en oändligt while loop där värdena från alla sensorer uppdateras kontinuerligt medan modulen väntar på avbrottet som orsakas via SPI när kommunikationsmodulen efterfrågar värden. Vid SPI avbrott, de senaste sensorvärdena hämtas och skickas vidare till kommunikationmodulen. Avbrottet aktiveras
som beskrivet i sektion \ref{sec:ctrl-int}

Halleffektsensorerna kommer skicka en avbrottssignal varje gång en magnet på
hjulet passerar en sensor. Avbrottsrutinen ser då till att öka den körda
distansen.

\end{document}
