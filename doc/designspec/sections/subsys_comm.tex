\documentclass[designspec/spec.tex]{subfiles}

\begin{document}

\section{Kommunikationsmodul}
Kommunikationsmodulen ska agera som taxins hjärna. Den styr bilen under autonom
körning och kommunicerar med fjärrklienten. Sensorvärden från sensormodulen
tolkas av kommunikationsmodulen som därefter skickar felvärden till
styrmodulen.

\subsection{Funktion}
Kommunikationsmodulen har tre huvudsakliga uppgifter; bildbehandling,
kommunicera med fjärrklienten och kontrollera taxin autonomt.

\paragraph{Bildbehandling} skall utföras av kommunikationsmodulen för att
avgöra bilens position i vägfilen och upptäcka stopplinjer. Med hjälp av
kamerans bilder på vägen skapas ett felvärde som kan användas för att justera
taxins riktning.

\paragraph{Kommunikation med fjärrklienten} sköts av kommunikationsmodulen för
att skicka sensorvärden och annan relevant information. Modulen skall även ta
emot ett uppdrag som har skapats utifrån en karta och destination som
användaren har matat in via fjärrklienten.

\paragraph{Autonomitet} utförs av kommunikationsmodulen för att utföra
uppdraget. Kommunikationsmodulen hämtar sensordata från sensormodulen och
bildbehandlar kamerabilderna för att utföra beslut i realtid. Besluten kommer
därefter att översättas till felvärden som skickas till styrmodulen.

\subsection{Hårdvaruimplementation} 
Kommunikationsmodulen kommer att implementeras med hjälp av en Raspberry Pi 3,
där en bluetooth och wifikomponent redan finns integrerat. Kameran kommer att
kopplas direkt till kameraporten som finns på Raspberry Pi-kortet.
GPIO-pinnarna kommer att användas för att ansluta till övriga moduler.
Eftersom SPI-protokollet ska användas, kommer GPIO-portar för de
signalerna som protokollet kräver användas. Det kommer att finnas
nivåskiftare som skiftar spänningen på signalerna som går mellan
kommunikationsmodulen och mikrokontrollerna.
% TODO hänvisa till kretsschema i bilaga

Kommunikationen med fjärrklienten kommer att ske med en inbyggd Wi-Fi eller
Bluetooth-modul på Raspberry Pi-enheten. Anslutningen kommer att ske via wifi
och skapas med en dator med operativsystemet Raspbian och program som
exempelvis wpa\_supplicant kan användas för att sätta upp anslutningen.

\subsubsection{Budget}
\begin{itemize}
    \item \textbf{Raspberry Pi 3} Modulens dator med inbyggt trådlöst
    nätverkskort.
    \item \textbf{Raspberry Pi Camera V2 Video Module} Kamera till Raspberry
    Pi.
    \item \textbf{2 $\times$ TXB0104 Level Shifter} Nivåskiftare för 4
    signaler.
\end{itemize}

\subsubsection{Kontroll och bedömning}
% TODO kontrollera pins samt gör bedömning av kommunikationsmodul
För att kunna köra systemet med SPI-protokollet kommer ett antal signaler att
behövas. De signaler som kommer att behövas är:
\begin{itemize}
    \item \textbf{MOSI} Data in till slav, data ut från master.
    \item \textbf{MISO} Data från slav, data in till master.
    \item \textbf{SCLK} En seriell klocka som är går ut från mastern.
    \item \textbf{SS1}  Skickas en aktivt låg signal vid selektion av
        slav.(Sensormodulen)
    \item \textbf{SS2}  Skickas en aktivt låg signal vid selektion av
        slav(Styrmodulen)
\end{itemize}

Raspberry Pi:n har stöd för 2 bussar med SPI-protokollet varav 10 pinnar
totalt. Eftersom vi bara kommer att använda 5 pinnar räcker antalet pinnar.
Upp till två nivåskiftare som konverterar mellan 5V och 3V3 kan användas då det
finns två 5V-pinnar samt två 3V3-pinnar.

\subsection{Mjukvaruimplementation}
Kommunikationsmodulen ska bestå av flera trådar TODO

\subsubsection{Avbrott}
% TODO avbrott för SPI

TODO komm mjukvara
%- skrivs i c eller möjligtvis c++
%- eventuell trådsuppdelning
%    - server för länk och uppdragshantering
%    - hämta från sensor och skicka till styr (avbrott?)
%    - bildhantering för att generera felvärde

\end{document}
