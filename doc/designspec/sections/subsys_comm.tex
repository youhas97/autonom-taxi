\documentclass[designspec/spec.tex]{subfiles}

\begin{document}

\section{Kommunikationsmodul}
Kommunikationsmodulen ska agera som taxins hjärna. Den styr bilen under autonom
körning och kommunicerar med fjärrklienten. Sensorvärden från sensormodulen
tolkas av kommunikationsmodulen som därefter skickar beslut till styrmodulen.

\subsection{Funktion}
Kommunikationsmodulen har tre huvudsakliga uppgifter; bildbehandling,
kommunicera med fjärrklienten och kontrollera taxin autonomt.

\paragraph{Bildbehandling} skall utföras av kommunikationsmodulen för att
avgöra bilens position i vägfilen och upptäcka stopplinjer. Med hjälp av
kamerans bilder på vägen skapas ett felvärde som kan användas för att justera
taxins riktning.

\paragraph{Kommunikation med fjärrklienten} sköts av kommunikationsmodulen för
att skicka sensorvärden och annan relevant information. Modulen skall även ta
emot en karta som användaren har matat in via fjärrklienten som kan användas
för den autonoma körningen.

\paragraph{Autonomitet} utförs av kommunikationsmodulen för att utföra
uppdraget. Kommunikationsmodulen hämtar sensordata från sensormodulen och
bildbehandlar kamerabilderna för att utföra beslut i realtid. Besluten kommer
därefter att utföras med kommandon som skickas till styrmodulen.

\subsection{Hårdvaruimplementation} 
\paragraph{Fjärranslutning}
Kommunikationen med fjärrklienten kommer att ske med en inbyggd wifi och
bluetooth-modul på raspberry-pi 3. Anslutningen kommer att ske via wifi och skapas med en dator
med operativsystemet linux och program som WPA-supplicant kan användas för att
sätta upp anslutningen.
\subsection{Mjukvaruimplementation}

\end{document}
