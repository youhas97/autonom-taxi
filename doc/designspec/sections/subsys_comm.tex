\documentclass[designspec/spec.tex]{subfiles}

\begin{document}

\section{Kommunikationsmodul}
Kommunikationsmodulen ska agera som taxins hjärna. Den styr bilen under autonom
körning och kommunicerar med fjärrklienten. Sensorvärden från sensormodulen
tolkas av kommunikationsmodulen som därefter skickar felvärden till
styrmodulen.

\subsection{Funktion}
Kommunikationsmodulen har tre huvudsakliga uppgifter; bildbehandling,
kommunicera med fjärrklienten och kontrollera taxin autonomt.

\paragraph{Bildbehandling} skall utföras av kommunikationsmodulen för att
avgöra bilens position i vägfilen och upptäcka stopplinjer. Med hjälp av
kamerans bilder på vägen skapas ett felvärde som kan användas för att justera
taxins riktning.

\paragraph{Kommunikation med fjärrklienten} sköts av kommunikationsmodulen för
att skicka sensorvärden och annan relevant information. Modulen skall även ta
emot ett uppdrag som har skapats utifrån en karta och destination som
användaren har matat in via fjärrklienten.

\paragraph{Autonomitet} utförs av kommunikationsmodulen för att utföra
uppdraget. Kommunikationsmodulen hämtar sensordata från sensormodulen och
bildbehandlar kamerabilderna för att utföra beslut i realtid. Besluten kommer
därefter att översättas till felvärden som skickas till styrmodulen.

\subsection{Hårdvaruimplementation} 
Kommunikationsmodulen kommer att implementeras med hjälp av en Raspberry Pi 3,
där en WLAN-komponent redan finns integrerat. Kameran kommer att kopplas direkt
till kameraporten som finns på Raspberry Pi-kortet. GPIO-pinnarna kommer att
användas för att ansluta till övriga moduler.  Eftersom SPI-protokollet ska
användas, kommer GPIO-portar för de signalerna som protokollet kräver användas.
Det kommer att finnas nivåskiftare som skiftar spänningen på signalerna som går
mellan kommunikationsmodulen och mikrokontrollerna. Ett kretsschema för modulen
finns i bilaga \ref{cdiag.comm}.

\subsubsection{Budget}
    Nedan komponenter i kommunikationsmodulen ingår inte i baschassit och
    behöver beställas.
\begin{itemize}
    \item \textbf{Raspberry Pi Camera V2 Video Module} Kamera till Raspberry
    Pi.
    \item \textbf{2 $\times$ \modShifter} Nivåskiftare för 4 signaler
    vardera.
\end{itemize}

\subsubsection{Kontroll och bedömning}
För att kunna köra systemet med SPI-protokollet kommer ett antal pinnar att
behövas. De signaler som kräver pinnar är:
\begin{itemize}
    \item \textbf{MOSI} Datasignal från master till slav.
    \item \textbf{MISO} Datasignal från slav till master.
    \item \textbf{SCLK} Synkron klocka från mastern till mikrokontrollerna.
    \item \textbf{SS1} Slave select för sensormodulen.
    \item \textbf{SS2} Slave select för styrmodulen.
    slav.
\end{itemize}
Raspberry Pi:n har stöd för 2 bussar med SPI-protokollet varav 10 pinnar
totalt. Eftersom endast 5 pinnar kommer användas räcker antalet pinnar.  Två
nivåskiftare som konverterar mellan 5V och 3V3 kan användas då det finns två
5V-pinnar samt två 3V3-pinnar.
% TODO/XXX lägga till dessa i listan? mer än 5 används väl då

Det mest krävande för kommunikationsmodulen är bildbehandlingen. Raspberry Pi
3 model B har en fyrakärnig BCM2837 klockad på 1.2GHz och 1GB internt minne.
Detta bör vara tillräckligt för att uppnå 20 bilder per sekund med en effektiv
implementation i C eller C++.

\subsection{Mjukvaruimplementation}
Raspberry:n kommer köra operativsystemet Raspbian och program som exempelvis
wpa\_supplicant för att sätta upp anslutningen för WLAN. Ett program för
kommunikationsmodulen kommer skrivas i C eller eventuellt C++ och ska bestå av
två trådar; en huvudtråd som hanterar uppdrag, bildbehandlar och svarar på
kommandon från fjärrklienten samt en tråd som sköter kommunikation med andra
moduler.

\paragraph{Huvudtråden} kommer att starta en TCP/IP-server för att lyssna på
kommandon från fjärrklienten. Tråden kommer köra en while loop där den först
accepterar inkommande anslutningar och agerar utefter protokollet som är
specificerat i sektion \ref{sec:wlproto}. Därefter kommer den hantera uppdraget
och eventuellt bildbehandla. Därefter uppdatera felvärde som ska skickas till
styrmodulen.

\paragraph{SPI-bussen} kommer hanteras av en dedikerad tråd. Den skall
kontinuerligt hämta värden från sensormodulen oberoende av andra trådar. När
bildbehandlingen är färdig med nästa bild skall felvärdet skickas till
styrmodulen. Detta sker i en separat tråd för att inte fördröja
bildbehandlingen i väntan på bussen.

\subsubsection{Uppdrag} \label{sec:comm-mission}
För att utföra uppdraget kommer en kö av kommandon tas emot från fjärrklienten.
När uppdraget börjar kör taxin framåt och följer filen med hjälp av
bildbehandling och reglering. Inför varje stopplinje tas ett kommando från kön
och utförs. Om ett hinder upptäcks under uppdraget kommer bilen att stanna
tills hindret har försvunnit eller eventuellt passera hindret. När ett visst
antal stopplinjer har passerats och kön är tom är uppdraget avklarat. Nedan är
en lista av alla möjliga kommandon hur de utförs.
\begin{labeling}{wwww}
    \item[\commIgnore] Kör förbi stopplinjen utan att stanna.
    \item[\commEnter] Sväng höger in i rondellen och hitta rondellens körfil.
    \item[\commExit] Sväng höger ut ur rondellen och hitta körfilen.
    \item[\commStop] Stanna framför stopplinjen, fortsätt efter några sekunder
    om kön inte är tom.
    \item[\commPark] Parkera i fickan till höger efter stopplinjen, lämna
    parkeringen och fortsätt efter några sekunder om kön inte är tom.
\end{labeling}

\end{document}
