\documentclass[designspec/spec.tex]{subfiles}

\begin{document}

\section{Kommunikationsmodul}
Kommunikationsmodulen ska agera som taxins hjärna. Den styr bilen under autonom
körning och kommunicerar med fjärrklienten. Sensorvärden från sensormodulen
tolkas av kommunikationsmodulen som därefter skickar beslut till styrmodulen.

\subsection{Funktion}
Kommunikationsmodulen har tre huvudsakliga uppgifter; bildbehandling,
kommunicera med fjärrklienten och kontrollera taxin autonomt.

\paragraph{Bildbehandling} skall utföras av kommunikationsmodulen för att
avgöra bilens position i vägfilen och upptäcka stopplinjer. Med hjälp av
kamerans bilder på vägen skapas ett felvärde som kan användas för att justera
taxins riktning.

\paragraph{Kommunikation med fjärrklienten} sköts av kommunikationsmodulen för
att skicka sensorvärden och annan relevant information. Modulen skall även ta
emot en karta som användaren har matat in via fjärrklienten som kan användas 
för den autonoma körningen.

\paragraph{Autonomitet} utförs av kommunikationsmodulen för att utföra
uppdraget. Kommunikationsmodulen hämtar sensordata från sensormodulen och
bildbehandlar kamerabilderna för att utföra beslut i realtid. Besluten kommer
därefter att utföras med kommandon som skickas till styrmodulen.

\subsection{Hårdvaruimplementation} 
Kommunikationsmodulen kommer att implementeras med hjälp av en Raspberry Pi 3,
där en bluetooth och wifikomponent redan finns integrerat. Kameran kommer att
kopplas direkt till kameraporten som finns på Raspberry Pi-kortet.
GPIO-pinnarna  kommer att användas för att ansluta till övriga moduler.
Eftersom {\iic}-protokollet kommer att användas, kommer endast GPIO2 och
GPIO3-pinnarna att användas eftersom de är data respektive klockan för de
pinnarna som stödjer {\iic}. Det kommer att finnas en nivåskiftare som klockan
och datan från GPIO pinnarna skiftar spänning i. Nivåskiftaren kommer att
försörjas med en spänning från en 3.3V pin, samt en 5V pin för att kunna skifta
spänningen från 3.3V som är den spänningen Raspberry Pi 3 stödjer, till 5V som
är den spänningen sensormodulen och styrmodulen använder.
% TODO hänvisa till kretsschema i bilaga

\paragraph{Fjärranslutning}
Kommunikationen med fjärrklienten kommer att ske med en inbyggd Wi-Fi eller
Bluetooth-modul på Raspberry Pi-enheten. Anslutningen kommer att ske via wifi
och skapas med en dator med operativsystemet Raspbian och program som
WPA-supplicant kan användas för att sätta upp anslutningen.

\subsection{Mjukvaruimplementation}

\end{document}
