\documentclass[designspec/spec.tex]{subfiles}

% Detta avsnitt bör innehålla:
%
%   -En beskrivning av hur processorerna kommunicerar med varandra (gäller även
%   kommunikation med PC:n om Blåtand används), protokoll och master/slave
%   förhållande.
%
%   -En beskrivning av vilken information (data, styrinformation, sensorvärden
%   etc.) som ska skickas mellan blocken, samt hur och vilken väg den ska
%   skickas. (Detaljer kring informationskodning och överföringsprotokoll kan
%   gruppen få utveckla under projektets gång).

\begin{document}

\section{Kommunikation}

\section{Kommunikationsmodul till styrmodul}
Kommunikationsmodulen ska skicka en hastighet $v$ och en styrradie $r$ till
styrmodulen med jämna mellanrum. Styrenheten ska försöka hålla hastighet och
radie som specificerat av det senaste mottagna värdet. Figur \ref{bf:comm-ctrl}
visar bitsekvensen för datan som sänds.

\begin{figure}[H]
    \centering
    \begin{bytefield}{16}
        \bitheader{0,7,8,15} \\
        \bitbox{8}{$v$}
        \bitbox{8}{$r$}
    \end{bytefield}
    \label{bf:comm-ctrl}
    \caption{Bitsekvensen för datan som skickas från kommunikationsmodulen till
    styrmodulen.}
\end{figure}

\paragraph{Hastigheten $v$} består av ett signerat 8-bitars heltal i
tvåkomplementsform. Värdet -128 motsvarar högsta fart bakåt, 0 motsvarar att
stå still och 127 motsvarar högsta fart framåt.

\paragraph{Styrradien $v$} består av ett signerat 8-bitars heltal i
tvåkomplementsform. Värdet -128 motsvarar maximal svängning åt vänster, 0
motsvarar raka hjul, 127 motsvarar maximal svängning åt höger.

\section{Sensormodul till kommunikationsmodul}

\end{document}
