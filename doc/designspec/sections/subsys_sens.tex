\documentclass[designspec/spec.tex]{subfiles}

\begin{document}

\section{Sensormodul}
Sensormodulen ska agera som taxins känselspröt. Den kommer att mäta av olika
värden från de olika sensorerna som finns för att sedan skicka mätdata till
kommunikationsmodulen.

\subsection{Funktion}
Sensormodulen ska hämta eventuellt filtrerade värden från sensorerna, omvandla
till linjära enheter och skicka vidare till kommunikationsmodulen.

\paragraph{Sensorer}
Taxin kommer att ha en optisk avståndsmätare som är riktad framåt för att
upptäcka hinder framför bilen samt en optisk avståndsmätare på sidan för att
detektera hinder höger om bilen, t.ex. vid omkörning.

\paragraph{Filtrering} Eftersom sensorerna kan bli påverkade av eventuella
störningar kommer det eventuellt att finnas brusfilter som filtrerar bort dessa
störningar för att skapa så noggranna sensorvärden som möjligt.

\subsection{Hårdvaruimplementation}
Sensormodulen består huvudsakligen av en mikrokontroller, och fyra sensorer; en
avståndsmätare på framsidan, en avståndsmätare på höger sida samt två
halleffektsensorer placerade på två av hjulen. Mikrokontrollern är klockad av 
en kristalloscillator. Kontrollern är även kopplad till en LCD-display
för att kunna visa värden vid felsökning.

På LCD:n har följande två pinnar valts till att ha ett konstant värde: R/W och
E. Eftersom inget krav finns på att kunna släcka LCD-panelen alternativt läsa från
displayen så är enable och write-signalen jordas.

Avståndsmätarna skickar en spänning direkt till mikrokontrollens AD-omvandlare.
Spänningens värde motsvarar ett avstånd. Utspänningen kan eventuellt behöva
filtreras från brus.

Önskad klockfrekvens från oscillatorn är 16Mhz.

Ett kretsschema över modulen finns i bilaga \ref{cdiag:sens}.

\subsubsection{Budget}
Nedan är produkter som ej medföljer baschassit och behöver beställas.
\begin{itemize}
	\item \textbf{\modMicrocontroller} Modulens microprocessor. 
    \item \textbf{\modDistf} Avståndsmätare för hinder framför bilen.
    \item \textbf{\modDists} Avståndsmätare för hinder höger om bilen.
    \item \textbf{\modLcd} LCD-display för att visa parametrar under körning
    i felsökningssyfte.
    \item \textbf{IQEXO3} Kristalloscillator på 16Mhz, skall använda som systemklocka till AVR.
\end{itemize}
Modulen använder även en {\modJtag} och tryckknapp som delas med styrmodulen.

\subsubsection{Kontroll och bedömning}
Nedan är följande pinnar som används på mikrokontrollern.
\begin{itemize}
   \item \textbf{PA0 \& PA2} Vardera port får insignal från en avståndsmätare.
   \item \textbf{PB2-PB3} Vardera port får en avbrottsignal från en odometer.
   \item \textbf{PB4-PB7} SS, MOSI, MISO och SCLK för SPI-bussen.
   \item \textbf{PC2-PC5} TCK, TMS, TDO, TDI för JTAG.
   \item \textbf{PC7} Register-select signal till LCD-display.
   \item \textbf{PD0-PD7} Databuss till LCD-display.
   \item \textbf{RESET(9)} Knapp till reset för MCU.
   \item \textbf{XTAL1} Klocka från Kristalloscillatorn.
\end{itemize}
Mikrokontrollen behöver endast kommunicera med kommunikationsmodulen via SPI
samt ta emot värden från sensorerna och göra enkla beräkningar vid omvandling.
Mikrokontrollen bör inte ha något problem att utföra dessa uppgifter.

\subsection{Mjukvaruimplementation} 
Sensormodulen ska bestå av en tom while loop och ett avbrott som aktiveras när
kommunikationsmodulen efterfrågar värden. Sensormodulen skickar då de senaste
sensorvärdena och samplar därefter nästa grupp av värden. Avbrottet aktiveras
som beskrivet i sektion \ref{sec:ctrl-int}

Halleffektsensorerna kommer skicka en avbrottssignal varje gång en magnet på
hjulet passerar en sensor. Avbrottsrutinen ser då till att öka den körda
distansen.

\end{document}
