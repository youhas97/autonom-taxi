\documentclass[designspec/spec.tex]{subfiles}

\begin{document}

\section{Fjärrklient}
Fjärrklienten fungerar som ett gränssnitt till användaren.

\subsection{Funktion}
Fjärrklientens uppgift är att bidra med ett gränssnitt till användaren för att
kontrollera och övervaka taxin.
TODO inmatning
TODO övervakning
TODO räkna karta
TODO skapa kommandolista utifrån uppdrag

\subsection{Mjukvaruimplementation}
För att hitta kortaste vägen ska kartan matas in i fjärrklienten för att med
hjälp av Dijkstras algoritm räkna ut kortaste vägen. Kartan
representeras med ett nätverk där varje nod motsvarar en tejpbit. Programmet
skrivs i Python.

\subsubsection{Algoritmer}
För att hitta kortaste vägen ska Dijkstras algoritm användas. Denna kommer
implementeras i Python.

\subsubsection{Representation av karta}

% datastrukturer
\begin{labeling}{wwww}
    \item[Nod] En nod är en klass som består av nodtyp och utgående bågar. Det
    finns tre olika nodtyper; stopplinje, parkeringsficka och rondell. Dessa
    representeras av siffrorna 1, 2 och 3, respektive.
    Grannar är alla de noder som objektet har en anslutande båge till. Avstånd
    anges för alla grannar.

    \item[Båge] En båge består av ett avstånd och en destination. Den
    representeras av en tuple där första värdet är ett avstånd och det andra
    värdet är en pekare till en nod.

    \item[Karta] Kartan utgörs av en lista med noder.
\end{labeling}

\end{document}
