\documentclass[designspec/spec.tex]{subfiles}

\begin{document}

\section{Styrmodul}
Styrmodulen är en modul som har i uppgift att ta emot styrsignaler ifrån
kommunikationsmodulen och skicka dessa till bilens drivmotor och svängmotor.

\subsection{Funktion}
Styrmodulens syfte är att hantera den direkta kontrollen av styrreglagen för
taxins motorer. Styrmodulen ska kontrollera både drivmotorn och svängmotorn
utifrån kommandon som ges av kommunikationsmodulen.

\subsection{Hårdvaruimplementation}

\subsubsection{Kopplingsschema}
TODO
\subsubsection{Budget}
Utöver motorn och servon som redan sitter monterade på chassit behöver
styrmodulen följande komponenter:

\begin{itemize}
	\item \textbf{ATMega16} Modulens microprocessor. 
	\item \textbf{JTAG} Finns för att kunna flasha microprocessorn. 
	\item \textbf{LCD-display} Ska visa parametrar under körning i
	felsökningssyfte.
\end{itemize}

\subsubsection{Kontroll}
Av portarna på microprocessorn används 7/8 av portA, 1/8 av portB, 6/8 av portC
och 5/8 av portD. D.v.s. portarna räcker till.
Se kopplingsschemat för att mer detaljerat se använda portar.

\subsubsection{Bedömning}

\subsection{Mjukvaruimplementation}

\subsubsection{Flödesschema}

\subsubsection{Avbrott}

\begin{itemize}
	\item \textbf{TWI\_Interrupt} Avbrottet sker när TWI-bussen adresserar
	styrmodulen. I avbrottet måste statusregistret TWSR maskas för att kunna
	avgöra vilken funktion som ska utföras. T.ex. skicka/ta emot data e.t.c. 

	\item \textbf{Output\_Compare\_Interrupt} Avbrottet sker när en PWM puls
	ska skickas ut till servon eller motorn. Avbrottet sker varje gång
	motsvarande timer-register har nått sitt maximum och startar om från noll.
\end{itemize}

\subsubsection{Huvud-loop}
I huvudloopen kommer värdena från kommunikationsmodulen göras om till
motsvarande tidsintervall i PWM. Beroende på hur snabbt det går så kommer
huvudloopen vänta på avbrott.


\subsection{Komponenter}
\paragraph{ATMega16} bla bla TODO
\paragraph{Drivmotor} bla bla TODO
\paragraph{Servo} bla bla TODO
\paragraph{LCD JM162A} bla bla TODO

\end{document}
