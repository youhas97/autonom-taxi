\documentclass[designspec/spec.tex]{subfiles}

\begin{document}

\section{Styrmodul}
Styrmodulen är en modul som har i uppgift att ta emot felvärden från
kommunikationsmodulen och reglera bilens drivmotor och svängmotor utifrån
dessa.

\subsection{Funktion}
Styrmodulens syfte är att hantera den direkta kontrollen av styrreglagen för
taxins motorer. Styrmodulen ska reglera drivmotorn och svängmotorn utifrån
felvärden som ges av kommunikationsmodulen.

\subsection{Hårdvaruimplementation}
Styrmodulen består av en mikrokontroller, taxins motorer samt en LCD för
felsökning. Ett detaljerat kretsschema finns i bilaga \ref{cdiag:ctrl}.

\subsubsection{Budget}
Utöver motorn och servon som redan sitter monterade på chassit behöver
styrmodulen följande komponenter.
\begin{itemize}
	\item \textbf{\modMicrocontroller} Modulens microprocessor. 
    \item \textbf{\modJtag} Debugger för programmering och felsökning med
        microprocessorn. 
    \item \textbf{\modLcd} LCD-display för att visa parametrar under körning i
        felsökningssyfte.
\end{itemize}

\subsubsection{Kontroll och bedömning}
Av portarna på microprocessorn används 8/8 av portA, 6/8 av portB, 4/8 av portC
och 1/8 av portD. D.v.s. portarna räcker till.

Mikrokontrollerna behöver endast utföra enkel reglering och därefter skapa en
PWM-signal för motorerna. Kommunikationen bestäms av kommunikationsmodulen
genom en "slave select", en signal som sätts till låg när kommunikationsmodulen
vill kommunicera med slaven.  Inget av detta är särskilt krävande och bör
klaras utan problem av ATMega1284.  Kontrollern har en flash-minne på 128kB,
vilket blr vara mer än tillräckligt för att lagra programmet.

PB5, även kallad MOSI, används för att ta emot data från kommunikationsmodulen
, gentemot PB6,även kallad MISO, som används för att skicka data till mastern.
PB4, även kallad CS (Chip Select) eller SS (Slave Select), sätts till låg av
kommunikationsmodulen för att indikera att kommunikation mellan styrmodulen och
kommunikationsmodulen.

\subsection{Mjukvaruimplementation}
Styrmodulens program kommer bestå av en main loop som reglerar bilens hastighet
efter ett felvärde. Ett avbrott som aktiveras när kommunikationsmodulen vill
kommunicera kommer att ta emot nya felvärden med jämna mellanrum.

\subsubsection{Huvud-loop}
I huvudloopen kommer värdena från kommunikationsmodulen göras om till
motsvarande tidsintervall i PWM. Beroende på hur snabbt det går så kommer
huvudloopen vänta på avbrott.

\subsubsection{Avbrott}
Så länge SS matas med en hög signal så kommer MISO att vara högohmig. Avbrottet
sker när SS matas med en låg signal. Mjukvaran kan uppdatera innehållet i SPDR
(SPI Data Register) i det här tillståndet, men data kommer inte att förskjutas
ut genom inkommande klockpulser innan SS matas med en hög signal. När en byte
har skickats så är överföringen slut och SPIF sätts. 

Om SPIE (SPI Interrupt Enable) i SPCR är satt, begärs ett nytt avbrott. Slaven
kan fortsätta placera ny data som ska skickas till SPDR innan avläsningen av
den inkommande datan sker. Den sista inkommande byten sparas i Buffer Register
för senare användning

\end{document}
