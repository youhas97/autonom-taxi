\documentclass[designspec/spec.tex]{subfiles}

\begin{document}

\section{Styrmodul}
Styrmodulen är en modul som har i uppgift att ta emot styrsignaler ifrån
kommunikationsmodulen och skicka dessa till bilens drivmotor och svängmotor.

\subsection{Funktion}

\subsection{Hårdvaruimplementation}
\subsubsection{Kopplingsschema}

\subsubsection{Budget}

\subsubsection{Kontroll}

\subsubsection{Bedömning}

\subsection{Mjukvaruimplementation}

\subsubsection{Flödesschema}

\subsubsection{Algoritmer}

\subsubsection{Datastrukturer}

\subsubsection{Avbrott}

\begin{itemize}
    \item \textbf{TWI\_Interrupt} Avbrottet sker när TWI-bussen adresserar
    styrmodulen. I avbrottet måste statusregistret TWSR maskas för att kunna
    avgöra vilken funktion som ska utföras. T.ex. skicka/ta emot data e.t.c. 

    \item \textbf{Output\_Compare\_Interrupt} Avbrottet sker när en PWM puls
    ska skickas ut till servon eller motorn. Avbrottet sker varje gång
    motsvarande timer-register har nått sitt maximum och startar om från noll. 
\end{itemize}
\subsubsection{Huvud-loop}

\subsection{Funktion}
Styrmodulens syfte är att hantera den direkta kontrollen av styrreglagen för
taxins motorer. Styrmodulen ska kontrollera både drivmotorn och svängmotorn
utifrån kommandon som ges av kommunikationsmodulen.

\subsection{Komponenter}
\paragraph{ATMega16} bla bla TODO
\paragraph{Drivmotor} bla bla TODO
\paragraph{Servo} bla bla TODO
\paragraph{LCD JM162A} bla bla TODO

\end{document}
