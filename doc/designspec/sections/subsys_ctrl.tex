\documentclass[designspec/spec.tex]{subfiles}

\begin{document}

\section{Styrmodul}
Styrmodulen är en modul som har i uppgift att ta emot felvärden från
kommunikationsmodulen och reglera bilens drivmotor och svängservo utifrån
dessa.

\subsection{Funktion}
Styrmodulens syfte är att hantera den direkta kontrollen av styrreglagen för
taxins motor. Styrmodulen ska reglera drivmotorn och svängservo utifrån
felvärden som ges av kommunikationsmodulen.

\subsection{Hårdvaruimplementation}
Styrmodulen består av en mikrokontroller, taxins motor samt en LCD för
felsökning. Mikrokontrollern är klockad av en kristalloscillator på 16 MHz. Ett
detaljerat kretsschema finns i bilaga \ref{cdiag:ctrl}.

\subsubsection{Budget}
Utöver motorn och servon som redan sitter monterade på chassit behöver
styrmodulen följande komponenter.
\begin{itemize}
	\item \textbf{\modMicrocontroller} Modulens microprocessor. 
    \item \textbf{\modJtag} Debugger för programmering och felsökning med
        microprocessorn. 
    \item \textbf{\modLcd} LCD-display för att visa parametrar under körning i
        felsökningssyfte.
    \item \textbf{IQEXO3} Kristalloscillator som systemklocka till AVR.
    \item \textbf{Tryckknapp} Knapp används till RESET.
\end{itemize}

\subsubsection{Kontroll och bedömning}
Nedan är följande pinnar som används på mikrokontrollern.
\begin{itemize}
   \item \textbf{PA4-PA7} Datasignaler till LCD (DB4-DB7).
   \item \textbf{PB0} RS-signal till LCD.
   \item \textbf{PB1} Enable-signal till LCD.
   \item \textbf{PB4-PB7} SS, MOSI, MISO och SCLK för SPI-bussen.
   \item \textbf{PC2-PC5} TCK, TMS, TDO, TDI för JTAG.
   \item \textbf{PD4} PWM-signal till drivmotor.
   \item \textbf{PD5} PWM-signal till svängservo.
   \item \textbf{RESET} Resetsignal.
   \item \textbf{XTAL1} Klocka från kristalloscillatorn.
\end{itemize}
Inga fler signaler behövs så pinnarna på mikrokontrollen är tillräckliga.

Mikrokontrollen behöver endast kommunicera via SPI, utföra enkel reglering och
därefter skapa en PWM-signal för motorn. Inget av detta är särskilt krävande
och bör klaras utan problem av ATMega1284. Kontrollern har ett flash-minne på
128kB, vilket bör vara mer än tillräckligt för att lagra programmet.

\subsection{Mjukvaruimplementation}
Styrmodulens program kommer bestå av en main loop som reglerar bilens hastighet
efter ett felvärde. Ett avbrott som aktiveras när kommunikationsmodulen vill
kommunicera kommer att ta emot nya felvärden med jämna mellanrum.

\subsubsection{Huvud-loop och reglering}
I huvudloopen kommer felvärdena från kommunikationsmodulen användas för att
reglera bilens hastighet och därefter skicka motsvarande PWM-signaler till
motorn och servon. Regleringen fungerar genom att räkna ut hastigheten eller
radien $v[n]$ med hjälp av ett felvärde $e[t]$ så att
\begin{equation*}
    v[n] = k_p \cdot e[n] + k_d \cdot (e[n]-e[n-1])
\end{equation*}
där $k_p$ och $k_d$ är konstanter som kan justeras experimentiellt. Den första
termen står för att justera utefter det nuvarande felet. Den andra termen står
för att justera utefter den nuvarande förändringen av felet.

\subsubsection{Avbrott} \label{sec:ctrl-int}
När SPI-kontrollern har tagit emot en byte av data sätter den SPIF i SPSR (SPI
Status Register) och aktiverar ett avbrott. Avbrottsrutinen kan kontrollera
flaggan för att avgöra att kommunikationsmodulen vill skicka felvärden och
börja ta emot värden.

\end{document}
