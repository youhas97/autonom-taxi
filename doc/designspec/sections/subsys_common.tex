\documentclass[designspec/spec.tex]{subfiles}

\begin{document}

\section{Gemensamma implementationer}

\subsection{Mjukvara}
För att hitta kortaste vägen ska kartan matas in i fjärrklienten för att med
hjälp av Dijkstras algoritm räkna ut kortaste vägen. Kartan representeras med
ett nätverk där varje nod motsvarar en tejpbit. Programmet skrivs i Python.

\subsubsection{Algoritmer}
För att hitta kortaste vägen ska Dijkstras algoritm användas. Denna kommer
implementeras i Python.

\subsection{Representation av karta}

\begin{labeling}{Datastrukturer}
    \item[Nod] En nod är en klass som består av nodtyp och utgående bågar. Det
    finns tre olika nodtyper; stopplinje, parkeringsficka och rondell.

    \item[Båge] En båge består av ett avstånd och en destination. Den
    representeras av en tuple där första värdet är ett avstånd och det andra
    värdet är en pekare till en nod.

    \item[Karta] Kartan utgörs av en lista med noder.
\end{labeling}

\end{document}
