\documentclass[designspec/spec.tex]{subfiles}

\begin{document}

\section{Gemensamma implementationer}

\subsection{Mjukvaruimplementation}
För att hitta kortaste vägen ska kartan matas in i fjärrklienten för att med
hjälp av Dijkstras algoritm räkna ut kortaste vägen. Kartan
representeras med ett nätverk där varje nod motsvarar en tejpbit. Programmet
skrivs i Python.

\subsubsection{Algoritmer}
För att hitta kortaste vägen ska Dijkstras algoritm användas. Denna kommer
implementeras i Python.

\subsection{Representation av karta}

\begin{labeling}{Nodtyper}

\item [Noder] 
Noder är en klass som består av nodtyp och vilka grannar den har.
Det finns tre olika nodtyper, stopplinje, parkingsficka och rondell. Dessa
representeras av siffrorna 1, 2 och 3, respektive.
Grannar är alla de noder som objektet har en anslutande båge till. Till alla
grannar anges avstånd.

\item[Karta]
Kartan utgörs av en lista med noder. 

\end{labeling}


\end{document}

