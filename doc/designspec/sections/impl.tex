\documentclass[designspec/spec.tex]{subfiles}

% Implementeringsstrategin kan innehålla svar på frågor som: 
%   -Skall konstruktionen ske ”utifrån och in” eller ”inifrån och ut”? 
%   -Hur kan man testa modulerna, integrera modulerna och testa hela systemet? 
%   -Hur kan man få feedback från systemet? 
%   -Hur ofta bör man sampla? 

\begin{document}

\section{Implementationsstrategi}

För att implementationen ska ske smidigt så ska en implementationsstrategi
bestämmas. Det inkluderar tillvägagångssätt, återkoppling och timing.
Tillvägagångssättet beskriver strategin för det praktiska, exempelvis i vilken
ordning modulerna ska implementeras. Återkoppling avser hur implementationen
kontinuerligt under arbetets ska testas, t.ex. genom att koppla en display till
en modul för att verifiera indata samt utdata.
Slutligen anger timing hur snabbt utdata respektive indatan ska
uppdateras i varje modul. Detta kan avse hur många bilder i sekunden
sensormodulen ska skicka till kommunikationsmodulen. 

\subsection{Tillvägagång}
Varje modul kommer att utvecklas parallellt för att sedan integreras
tillsammans. Eftersom varje modul kopplas till kommunikationsmodulen kan en
modul delvis integreras så fort den är klar om kommunikationsmodulen är
redo. För att i så hög grad som möjligt undvika stora problem i slutet ska
funktioner som beror av varandra och som finns i olika moduler implementeras
samtidigt för att sedan testas. Ett exempel är avståndsigenkänningen i
kommunikationsmodulen och stoppfunktionen i styrmodulen. Där båda kan testas
tillsammans för att verifiera att bilen kan stanna för ett hinder.
% TODO utveckla

\subsection{Återkoppling}
För att kunna felsöka problem under implementationen kommer olika sätt att få
återkoppling att användas. Sensor- och styrmodulen ska kopplas till en
LCD-display för att kunna visa värden. Värden kan också skickas via bussen till
kommunikationsmodulen när den är implementerad. Kommunikationsmodulen kommer
köra en SSH-server så att det går att komma åt värden på modulen. Värden kommer
kunna skrivas till \mono{stdout} eller filer. Programmet kommer också kunna
köras via GDB så att minnet direkt kan läsas. På liknande sätt kan
fjärrklienten på datorn felsökas.

\subsection{Timings}
Det som begränsar frekvensen av beslut från kommunikationsmodulen är
bildbehandlingen. Det vore önskvärt att nå åtminstone 20 bilder sekund. Utifrån
vilken bildbehandlingsfrekvens som uppnås kommer frekvensen av sampling från
sensorerna och styrkommandon till styrmodulen att anpassas. 

\end{document}
