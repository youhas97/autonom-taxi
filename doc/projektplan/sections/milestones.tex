\documentclass[projektplan/plan.tex]{subfiles}

\newcounter{milestoneNo} % nummer för varje milstolpe

\newenvironment{milestonelist}
{
    \setlength{\tabcolsep}{12pt}
    \renewcommand{\arraystretch}{1.6}
    \begin{longtable}{p{8mm}p{75mm}p{25mm}}
        \bfseries Nr &
        \bfseries Beskrivning &
        \bfseries Datum 
	\\\hline\endhead
}{
    \end{longtable}
}

\newcommand{\milestone}[2]{
    \stepcounter{milestoneNo}
    %nr            % beskr  % datum 
    \arabic{milestoneNo} & #1     & #2 \\
}

\begin{document}

\section{Milstolpar och beslutspunkter}
\subsection{Milstolpar}
\begin{milestonelist}
\milestone{Designspecifikationen är klar}{2018-11-06}
\milestone{Diskreta komponenter kopplade}{2018-09-30}
\milestone{Verifiering av elektronik och att bussarna fungerar}{2018-10-31}
\milestone{Utvecklingsmiljöer för processor och mikrokontroller
   är uppsatta}{2018-10-31}
\milestone{Kommunikation mellan moduler och interna mikrokontroller samt
    processor fungerar som den ska.}{2018-10-31}
\milestone{Bil vid dennna milstolpe ska kunna åka i alla
    riktningar}{2018-10-31}
\milestone{Bilen ska vid det här läget kunna fjärrstyras.}{2018-10-31}
\milestone{Kameran ska ta bilder och bilderna behandlas för att känna igen
    skelett}{2018-10-31}
\milestone{En lämplig datastruktur för representation av kartan och dess noder
    ska vara färdig. Man ska mha datorn kunna bygga en egen karta, helst
    användarvänligt.}{2018-10-31}
\milestone{Mätvärden skickas till datorn som representeras i SI-enhet i valfri
    användargränssnitt.}{2018-10-31}
\milestone{Representation av kartan kan skickas till systemet.}{2018-10-31}
\milestone{Någon form av PD-reglering infört.}{2018-10-31}
\milestone{Reglerparameterna till PD-reglering kan skickas
    fortlöpande.}{2018-10-31}
\milestone{Bilen kan hålla sig inom vägfilen}{2018-10-31}
\milestone{Bilen kan åka i rondeller}{2018-10-31}
\milestone{Bil åker tillbaka till startposition}{2018-10-31}
\milestone{Stannar inför hinder}{2018-10-31}
\milestone{Bilen klarar av att parkera i fickan.}{2018-10-31}
\milestone{Bilen tar kortaste vägen till sina mål}{2018-10-31}
\milestone{Bilen undviker hinder genom att på ett säkert sätt åka runt
    hindret.}{2018-10-31}
\milestone{Bilen klarar av uppdraget som definierad enligt banspecifikationen
    tre av fyra försök}{2018-10-31}
\end{milestonelist}

\subsection{Beslutspunkter}
En beslutspunkt är ett moment där projektgruppen har ett kort möte med
beställaren. Projektledaren ser till att de dokument som är fundamenten för
respektive beslut är färdiga och granskade. Dokumenten ska vara distribuerade
till vederbörande parter i god tid innan mötet. På mötet följs mallen för
beslutspunkten och fungerar som en checklista. Mallen kan sedan sparas som ett
beslutsprotokoll.

\subsubsection*{BP0} Vid beslutspunkt 0 godkänns projektdirektivet vilket t.ex. innebär att
projektet ska göras, medel ska tillelas projektgruppen för att de förstudier
som ska göras till  BP1 blir utförda.

\subsubsection*{BP1} 
Vid beslutspunkt 1 har förstudierna utförts och det som ska göras är specificerat. Det
finns också en stadig plan för hur projektet skall fortsätta, inte minst fram
till BP2. Beställaren bestämmer i BP1 om projektgruppen kommer att fortsätta
med förberedelser inför utförandefasen. Här ska kravspecifikationen samt en
preliminär projektplan vara färdigt.

\subsubsection*{BP2}
Vid beslutspunkt 2 skall beslut tas om att påbörja projektets utförandefas. Detta är ett
essentiellt beslut då projektet fortsättningsvis kommer att kosta pengar/tid.
Många involveras och de komponenter som krävs för att utföra projekter kommer
att köpas in. De dokument som ska finnas vid BP2 är en godkänd
kravspecifikation, systemskiss, projektplan och ett kundkontrakt.

\subsubsection*{BP3}
Vid beslutspunkt 3 ska en mer noggrann uppskattning av arbetet i projektet
göras. Beslutspunkten hålls efter designfasen då man specificerat exakt hur
alla moment ska geomföras och hur allt ska konstrueras. Dokumenten som ska vara
färdiga är designspecifikationen och eventuellt en modifierad projektplan.

\subsubsection*{BP4}
Vid beslutspunkt 4 genomförs extra genomgång av utförandefasen (behövs inte i
detta projekt). Produkten och
arbetets resultat/kvalitet granskas. Beslutspunkten är placerad så att ett
antal delmoment har uppnåtts och testats. Här ska designspecifikationen och
testprotokoll vara färdigt.

\subsubsection*{BP5}
Vid beslutspunkt 5 avgörs om projektet har nått målen och om resultatet ska
användas.Arbetet skall vara utfört som förväntat, produkten ska hålla rätt
kvalitet, alla dokument ska finnas. Här ska leveransen kontrolleras för att man
ska kunna avgöra om den är förberedd för leverans.

\subsubsection*{BP6}
Vid beslutspunkt 6 beslutar man om projektet ska avslutas. Här avgörs om
projektgruppen ska upplösas och om resuserna ska lämnas tillbaka. Här godkänns
också leveransen.

    \renewcommand{\arraystretch}{1.6}
    \begin{longtable}{p{8mm}p{75mm}p{25mm}}
        \bfseries BP &
        \bfseries Beskrivning &
        \bfseries Datum 
	\\\hline\endhead
    BP0 & Godkännande av projektdirektiv, start av förstudier & 2018-09-07 \\
    BP1 & Godkännande av kravspecifikation, start av förberedelsefasen &
    2018-09-18 \\
   BP2 & Godkännande av projektplanering, start av utförandefasen & 2018-10-04
    \\
    BP3 & Godkännande av designspecifikation, fortsättning av utförandefasen &
    2018-11-09\\
    BP4 & Används ej & \\
    BP5 & Godkännande av produktens funktionalitet, leverans & 2018-12-13 \\
    BP6 & Godkännande av leverans, upplösning av projektgrupp & 2018-12-21 \\

    \end{longtable}

\newpage
\end{document}
