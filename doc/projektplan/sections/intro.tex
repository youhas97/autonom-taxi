\documentclass[projektplan/plan.tex]{subfiles}

\begin{document}
\section{Beställare}
Avdelnigen ISY hos tekniska högskolan vid Linköpings universitet vill undersöka möjligheterna
att konstruera en autonom taxibil. ISY önskar beställa ett antal prototyper som ska delta i en
tävling där konstruktionsalternativen kommer utvärderas. Därmed har Mattias Kryssander beställt 
en prototyp varpå en kravspecifikation har ställts i dialog med beställaren Mattias Kryssander.
\section{Översiktlig beskrivning av projektet}
\subsection{Syfte och mål}
Målet är att konstruera en autonom taxibil som ska kunna köra i en bana och
plocka upp samt lämna av passagerare. I banan kan det förekomma hinder av olika
slag. Syftet med uppgiften är att fördjupa projektgruppens kunskaper inom
elektronik, programmering och framför allt samarbete inom en projektgrupp.

Projektgruppens mål är att leverera en färdig produkt innan sista leveransdatum
samt uppfylla alla baskrav i den inlämnade kravspecifikationen.
\subsection{Leveranser}
\label{sec:doc}
Nedan följer en lista av de dokument som skall medfölja produkten vid leverans.
{\renewcommand{\arraystretch}{1.6}
\begin{longtable}{p{4.5cm}p{1.5cm}p{5cm}p{2.2cm}p{1.2cm}}
    \bfseries Dokument &
    \bfseries Språk &
    \bfseries Syfte &
    \bfseries Målgrupp &
    \bfseries Format \\\hline
    Teknisk dokumentation &
    Svenska &
    Ge konstruktionsunderlag, samt dokumentation för underhåll och
    felsökning. &
    Beställare &
    PDF
    \\
    Användarhandledning &
    Svenska &
    Ge beskrivning av produktens användning. &
    Beställare, användare &
    PDF
    \\
    Designspecifikation &
    Svenska &
    Beskriva implementeringen av hårdvaran samt mjukvaran till systemet. &
    Underhållare, utvecklare &
    PDF
    \\
    Projektplan &
    Svenska &
    Beskriva hur projektet kommer att utföras och vilka strategier som
    tillämpas &
    Beställare &
    PDF
    \\
    Tidplan &
    Svenska &
    Planera hur arbetet ska fördelas och när det ska utföras. &
    Löne\-avdelning, projekt\-ledaren &
    XLS
    \\
    Systemskiss &
    Svenska &
    En skiss över implementationen av systemet &
    Beställare, utvecklare &
    PDF
    \\
    
    \endhead
\end{longtable}}

\noindent


\subsection{Begränsningar}
Den konstruerade bilen behöver ej ta hänsysn till små väghinder på banan, så
som låga vägkanter eller gropar, då hindren är förutbestämda och specifierade
i banspecifikationen. Dessutom behöver bilen inte detektera parkeringsfickor
eller rondeller via bildbearbetning, då en banspecifikationen kräver att olika
färger färger av tejp indikerar olika hinder på banan.

Projektgruppen behöver inte heller konstruera en bil från grunden. Ett färdigt
chassi med färdigmonterade och färdigkopplade motorer samt en färdigmonterad
och färdigkopplad mikrodator med USB-portar finns ursprungligen.

\end{document}
