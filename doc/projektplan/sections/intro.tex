\documentclass[projektplan/plan.tex]{subfiles}

\begin{document}
\section{Beställare}

\section{Översiktlig beskrivning av projektet}
\subsection{Syfte och mål}
Målet är att konstruera en autonom taxibil som ska kunna köra i en bana och
plocka upp samt lämna av passagerare. I banan kan det förekomma hinder av olika
slag. Syftet med uppgiften är att fördjupa projektgruppens kunskaper inom
elektronik, programmering och framför allt samarbete inom en projektgrupp.

Projektgruppens mål är att leverera en färdig produkt innan sista leveransdatum
samt uppfylla alla baskrav i den inlämnade kravspecifikationen.
\subsection{Leveranser}
\subsection{Begränsningar}
Den konstruerade bilen behöver ej ta hänsysn till små väghinder på banan, så
som låga vägkanter eller gropar, då hindren är förutbestämda och specifierade
i banspecifikationen. Dessutom behöver bilen inte detektera parkeringsfickor
eller rondeller via bildbearbetning, då en banspecifikationen kräver att olika
färger färger av tejp indikerar olika hinder på banan.

Projektgruppen behöver inte heller konstruera en bil från grunden. Ett färdigt
chassi med färdigmonterade och färdigkopplade motorer samt en färdigmonterad
och färdigkopplad mikrodator med USB-portar finns ursprungligen.

\end{document}

