\documentclass[projektplan/plan.tex]{subfiles}

\begin{document}
\section{Beställare}
Avdelnigen ISY på tekniska högskolan vid Linköpings universitet vill undersöka
möjligheterna att konstruera en autonom taxibil. ISY önskar beställa ett antal
prototyper som ska delta i en tävling där konstruktionsalternativen kommer
utvärderas. Mattias Kryssander är beställare av en prototyp enligt en
kravspecifikation som ställts i dialog med beställaren.

\section{Översiktlig beskrivning av projektet}
\subsection{Syfte och mål}
Målet är att konstruera en autonom taxibil som ska kunna köra i en bana och
plocka upp samt lämna av passagerare. I banan kan det förekomma hinder av olika
slag. Syftet med uppgiften är att fördjupa kunskaper inom elektronik,
programmering och framför allt samarbete inom en projektgrupp.

Projektgruppens mål är att leverera en färdig produkt innan sista leveransdatum
samt uppfylla alla baskrav i kravspecifikationen.

\noindent
\begin{minipage}{\textwidth}
\subsection{Leveranser}
\label{sec:doc}
Nedan följer en lista av de leveranserna som ska ske under projekten.
{\renewcommand{\arraystretch}{1.6}
\begin{longtable} {p{7cm}p{3cm}p{5cm}}

    \bfseries Leverans &
    \bfseries Levereras till &
    \bfseries Datum  \\\hline 
    Tidsrapporter &
    Beställare &
    Från om med 45 till och med vecka 51 2018
    \\
    Statusrapporter &
    Beställare &
    Vid beställares begäran
    \\
    Projektplan &
    Beställare &
    Senast 4 november 2018
    \\
    Tidsplan &
    Beställare &
    Senast 4 november 2018
    \\
    Systemskiss  &
    Beställare &
    Senast 4 november 2018
    \\
    Designspecifikation (första version) &
    Handledare &
    Senast 6 november 2018
    \\
    Designspecifikation (slutgiltig version) &
    Handledare &
    Senast 9 november 2018
    \\
    Teknisk dokumentation &
    Beställare &
    Senast 13 november 2018
    \\
    Användarhandledning &
    Beställare &
    Senast 13 november 2018
    \\
    Efterstudie för projekt &
    Beställare &
    Senast 20 december 2018
    \\ 
    Produkt &
    Beställare &
    Vecka 51
    \\   
    Återlämning av hårdvara och utrustning &
    Beställare &
    Senast 21 december 2018
    \\

    \endhead
\end{longtable}}
\end{minipage}

\subsection{Begränsningar}
Den konstruerade bilen behöver inte ta hänsysn till små väghinder på banan,
såsom låga vägkanter eller gropar, då dessa är förutbestämda och specifierade i
banspecifikationen. Dessutom behöver bilen inte detektera parkeringsfickor
eller rondeller via bildbearbetning, då banspecifikationen kräver att olika
färger av tejp indikerar olika hinder på banan.

Projektgruppen behöver inte heller konstruera en bil från grunden. Ett färdigt
chassi med färdigmonterad/färdigkopplad motor och mikrodator med
USB-portar finns ursprungligen. 

\end{document}
