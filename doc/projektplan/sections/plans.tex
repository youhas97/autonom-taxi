\documentclass[projektplan/plan.tex]{subfiles}

\begin{document}

\section{Tidplan}

\section{Förändringsplan}

\section{Kvalitetsplan}
För att bibehålla en hög kvalitet ska material såsom kod, dokument,
presentationer etc. granskas av någon eller några av medlemmarna i gruppen.
Även testning av mjukvara och hårdvara ska utföras.

\subsection{Granskningar}
Allt material som produceras av gruppen ska granskas av samtliga medlemmar.
Detta för att andra medlemmar kan ha åsikter på förbättringar som författaren
inte tänkt på eller inte har kunskap om. På så sätt fås en högre kvalitet. I
första hand ska den medlem som har mest kunskap inom det område granska.

\subsubsection{Kod}	
Granskning av kod kan innefatta optimering såsom minska redundans samt
effektivisera. Det kan innefatta att skriva om så att kod kan skrivas på ett
kortare sätt eller med modulärt.

\subsubsection{Dokument}	
Genom att granska dokument kan fel såsom stavfel, formuleringar och
meningsuppbyggnader rättas till. Detta är speciellt önskvärt då en utomstående
kan upptäcka fel som författaren kan ha svårt att hitta.

\subsubsection{Presentationer}	
Presentationer granskas för att säkerhetsställa att inga felaktigheter i fakta,
utförande eller ambitioner presenteras.

\subsubsection{Schema}	
Schemat ska granskas av alla medlemmar för att kunna anpassas efter allas
scheman. Detta för att se till att så många som möjligt kan närvara vid
samtliga tillfällen. 

\subsection{Testplan}
För att testa mjukvaran ska testprogram skrivas som matar in indata och
förväntar sig en viss utdata. Skiljer sig förväntad utdata från faktiskt data
ses detta som ett fel. Varje modul kommer testas enskilt för att sedan testa
kommunikation mellan dessa. Testning av hårdvara sker genom att se till att de
olika hårdvarukomponenter kan kommunicera sinsemellan samt att de beter sig som
annonserad.

När både mjuk- och hårdvara har testats och det har verifieras att dessa
fungerar separat ska de testas tillsammans. Tills konstruktionen har
färdigställts går det inte har sätta upp när dessa tester ska utföras men
testerna kan ändå definieras.

\vspace{5mm}
\noindent
Tester som ska utföras:
\begin{itemize}
    \item Röra sig korrekt när den detekterar ett streck i en viss färg.
    \item Parkera i en parkeringsficka.
    \item Kan åka framåt, bakåt och åt båda sidorna.
    \item Kartan kan presenteras korrekt utifrån den data som läses in.
    \item Åka runt banan utan att stanna.
    \item Hålla sig inom en fil.
    \item Undvika hinder.
    \item Testa kommunikation mellan modulerna.
    \item Få kontakt med bilen från datorn.
\end{itemize}

%Riskanalys, not necessary
%\section{Riskanalys}

\section{Prioriteringar}
kill me

\section{Projektavslut}
fill me

\end{document}
