\documentclass[projektplan/plan.tex]{subfiles}

% tabell för aktivitet
\newcounter{activitycatno} % nummer för kategori
\newcounter{activityno}[activitycatno] % nummer för varje aktivitet
\renewcommand\theactivityno{\theactivitycatno.\arabic{activityno}}

\newenvironment{activitylist}
{
    \setlength{\tabcolsep}{12pt}
    \renewcommand{\arraystretch}{1.6}
    \begin{longtable}{p{18mm}p{25mm}p{65mm}p{6mm}p{7mm}}
        \bfseries Nr &
        \bfseries Aktivitet &
        \bfseries Beskrivning &
        \bfseries Tid &
        \bfseries Ber.
	\\\hline\endhead
}{
    \end{longtable}
}

\newcommand{\activity}[5]{
    \refstepcounter{activityno}\label{act:#1}
    %nr            % akt    % beskr % tid   % beroenden
    \theactivityno & #2     & #5    & #3    & #4 \\
}
\newcommand{\actcat}[1]{
    \hline\multicolumn{5}{c}{#1}\refstepcounter{activitycatno}\\\hline}

\newcommand{\arefproc}[1]{\ref{act:#1} }
\NewDocumentCommand\aref{>{\SplitList{,}}m}{\ProcessList{#1}{\arefproc}}

\begin{document}

\section{Aktiviteter}
Nedan följer en lista av alla de aktiviter som hela projektet har delats upp i.
Aktiviteter är uppdelade i olika kategorier och är numrerade efter kategori.
Tidskolumnent visar en uppskattning av antalet timmar som aktiviteten förväntas
att ta. Kolumnen för beroenden visar vilka andra aktiviteter som måste vara
avklarade innan en aktivitet kan bli avklarad.

\begin{activitylist}
    \actcat{Kommunikationsmodul}
    \activity{comm:cam}{Kamera}{5}{}{Installera kamera och gör dess bilder
    bearbetningsbara.}
    \activity{comm:reg}{PD-reglering.}{3}{\aref{ip:err,ctrl:drive,ctrl:steer}}
    {Reglera styrning utefter felvärde.}
    \activity{comm:map}{Tolka karta}{8}{\aref{ui:map-in}}{Tolka inmatad karta.}
    \activity{comm:map}{Bestäm position.}{4}{\aref{comm:map}}{Bestäm taxins
    position från i tolkad karta.}

    \actcat{Bildbearbetning}
    \activity{ip:edge}{Upptäck kanter.}{10}{\aref{comm:cam}}{Upptäck
    kanter i bilder.}
    \activity{ip:line}{Upptäck linjer.}{10}{\aref{ip:edge}}{Upptäck
    linjemarkeringar på vägen.}
    \activity{ip:2d}{Transformera linjer till
    2D.}{10}{\ref{act:ip:line}}{Transformera 3d-perspektiv till en ortografisk
    2d-bild sedd uppifrån.}
    \activity{ip:err}{Avgör felvärde.}{10}{\aref{ip:2d}}{Beräkna
    felvärde från 2D karta.}

    \actcat{Styrmodul}
    \activity{ctrl:drive}{Drivmotor}{2}{}{Reglera taxins fart genom att
    kontrollera drivmotorn utifrån kommunikationsmodulens kommandon.}
    \activity{ctrl:steer}{Svängmotor}{2}{}{Reglera svängradien genom att
    kontrollera svängmotorn utifrån kommandon.}
    \activity{ctrl:break}{Broms}{2}{\aref{ctrl:drive}}{Se till att bilen
    kan bromsa.}

    \actcat{Sensormodul}
    \activity{sens:filter}{Filtrera brus}{4}{}{Filtrera brus från analoga
    sensorer.}
    \activity{sens:sens}{Installera sensorer.}{10}{}{Koppla sensorer och gör
    dess värden tillgängliga.}
    \activity{sens:conv}{Konvertera värden}{3}{\aref{sens:sens}}{Konvertera
    sensorvärden till SI-enheter.}
    \activity{sens:send}{Skicka värden}{5}{\aref{sens:conv}}{Skicka värden till
    kommunikationsmodulen.}
    \activity{sens:lcd}{LCD-display}{4}{\aref{sens:sens}}{Visa utvalda värden
    på LCD-display.}

    \actcat{Användargränssnitt}
    \activity{ui:rc}{Styra taxi}{10}{}{Kör och styr taxin från gränssnittet.}
    \activity{ui:param}{Mata in parametrar}{2}{}{Mata in konstantparametrer
    under körning.}
    \activity{ui:data}{Visa mätdata}{2}{}{Visa upp mätdata från taxin.}
    \activity{ui:map-in}{Mata in karta}{10}{}{Mata in en karta av banan.}
    \activity{ui:map}{Rita karta}{10}{}{Rita en karta på gränssnittet med
    taxins nuvarande position markerad.}

    \actcat{Anslutningar}
    \activity{io:voltage}{Konvertera spänning}{5}{}{Se till att Raspberry Pi:n
    kan kommunicera via 5V-buss.}
    \activity{io:wireless}{Trådlös länk}{8}{}{Implementera en trådlös länk
    mellan kommunikationsmodulen och fjärrklienten.}

    \actcat{Utbildning}
    \activity{edu:opencv}{OpenCV}{3}{}{Förstå hur OpenCV fungerar.}
    \activity{edu:avr}{AVR}{3}{}{Förstå arkitektur och I/O för
    AVR-processorer.}
    \activity{edu:proto}{Överförings\-protokoll}{3}{}{Undersök protokoll som
    I$^2$C, SPI, UART m.m.}
    \activity{edu:kicad}{KiCad}{3}{}{Lära sig använda KiCad.}

    \actcat{Dokument}
    \activity{doc:designspec}{Designspe\-cifikation}{50}{}{Skapa detaljerad
    design demonstrerad i en specifikation.}
    \activity{doc:tecdoc}{Teknisk dokumentation}{50}{}{Dokumentera
    implementationen.}
    \activity{doc:study}{Efterstudie}{30}{}{Utför en efterstudie.}
    \activity{doc:manual}{Användar\-manual}{20}{}{Skriv en användarmanual.}
    \activity{doc:time}{Tidsrapporter}{10}{}{Rapportera tid varje vecka.}
    \activity{doc:status}{Statusrapporter}{5}{}{Rapportera status vid begäran.}
\end{activitylist}

\newpage

\end{document}
