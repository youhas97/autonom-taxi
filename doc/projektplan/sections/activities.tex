\documentclass[projektplan/plan.tex]{subfiles}

% tabell för aktivitet
\newcounter{activityno}

\newenvironment{activitylist}
{
    \setlength{\tabcolsep}{12pt}
    \renewcommand{\arraystretch}{1.6}
    \begin{longtable}{p{18mm}p{25mm}p{65mm}p{6mm}p{5mm}}
        \bfseries Nr &
        \bfseries Aktivitet &
        \bfseries Beskrivning &
        \bfseries Tid &
        \bfseries Ber.
	\\\hline\endhead
}{
    \end{longtable}
}

\newcommand{\activity}[5]{
    \refstepcounter{activityno}\label{act:#1}
    %nr            % akt    % beskr % tid   % beroenden
    \theactivityno & #2     & #5    & #3    & #4 \\
}
\newcommand{\actcat}[1]{
    \hline\multicolumn{5}{c}{#1}\\\hline}

\newcommand{\arefproc}[1]{\ref{act:#1} }
\NewDocumentCommand\aref{>{\SplitList{,}}m}{\ProcessList{#1}{\arefproc}}

\begin{document}

\section{Aktiviteter}
Nedan följer listor över alla de aktiviter som projektet har delats upp i.
Aktiviteter är uppdelade i olika kategorier och är numrerade efter kategori.
Tidskolumnen visar en uppskattning av antalet timmar som aktiviteten förväntas
att ta. Kolumnen för beroenden visar vilka andra aktiviteter som måste vara
avklarade innan en aktivitet kan bli avklarad.

\newpage
\subsection{Designfas}
Nedan följer de aktiviteter som är en del av designfasen. Dessa aktiviteter,
delar av allmänna aktiviter samt en buffer summerar till 420 inplanerade timmar
under designfasen.

\begin{activitylist}
    \actcat{Design}
    \activity{des:hwsys}{Systemhårdvara}{45}{}{Designa sammankopplingen mellan
    alla moduler.}
    \activity{des:hwmod}{Modulshårdvara}{75}{}{Designa hårdvaran av
    kommunikationsmodulen, styrmodulen och sensormodulen.}
    \activity{des:sw}{Mjukvara}{65}{}{Designa strukturerna för fjärrklientens,
    kommunikationsmodulens, styrmodulens och sensormodulens program.}

    \actcat{Dokument}
    \activity{doc:report}{Rapporter}{75}{}{Skriva, formatera och sammanställa
    rapporter.}
    \activity{doc:bdiag}{Rita blockdiagram}{45}{}{Rita blockdiagram över hela systemet,
    kommunikationsmodulen, styrmodulen och sensormodulen.}
    \activity{doc:cdiag}{Rita kretsscheman}{45}{}{Rita kretsschema över hela
    systemet, kommunikationsmodulen, styrmodulen och sensormodulen.}

    \actcat{Övrigt}
    \activity{des:meet}{Möte}{40}{}{Genomföra möten.}
\end{activitylist}

\newpage
\subsection{Utförandefas}
Utförandefasen består av nedanstående aktiviteter och delar av allmänna aktiviteter.
Dessa aktiviteter samt en buffer summerar till 700 inplanerade timmar.
\begin{activitylist}
    \actcat{Allmänt}
    \activity{all:install}{Koppla och installera}{50}{}{Konstruera hårdvaran av
    hela systemet, kommunikationsmodulen, styrmodulen och sensormodulen.}
    \activity{all:sw}{Implementera mjukvarubaser}{60}{}{Implementera
    mjukvarubaser som körs på fjärrklienten, kommunikationsmodulen, styrmodulen
    och sensormodulen.}

    \actcat{Kommunikationsmodul}
    \activity{comm:sw}{Mjukvara}{13}{\aref{all:sw}}{Implementera mjukvara på
    kommunikationsmodul utöver bildbearbetning och autonom körning. Bland annat
    göra kamerans bilder bearbetningsbara samt skicka och ta emot
    fjärrstyrningskommandon.}

    \actcat{Bildbearbetning}
    \activity{ip:edge}{Upptäck kanter.}{15}{\aref{all:install}}{Upptäck kanter
    i bilder.}
    \activity{ip:line}{Upptäck linjer.}{15}{\aref{ip:edge}}{Upptäck
    linjemarkeringar på vägen.}
    \activity{ip:2d}{Transformera linjer.}{15}{\ref{act:ip:line}}{Transformera
    perspektivbild av linjer till en ortografisk projektion sedd uppifrån.}
    \activity{ip:err}{Avgör felvärde.}{15}{\aref{ip:2d}}{Beräkna felvärde från
    linjekarta.}

    \actcat{Autonom körning}
    \activity{sdc:gps}{Avgör position.}{8}{\aref{all:sw}}{Tolka karta och
    bestäm taxins position i kartan.}
    \activity{sdc:tsp}{Hitta kortaste väg}{10}{}{Implementera algoritm för
    taxin att hitta kortaste vägen mellan två stopplinjer i karta.}
    \activity{sdc:reg}{Följ
    vägfil}{10}{\aref{all:sw,ip:err,ctrl:drive}}{Reglera styrning utefter
    felvärde och följ höger vägfil, inklusive rondeller.}
    \activity{sdc:nav}{Navigera}{20}{\aref{sdc:gps,sdc:reg}}{Navigera till
    godtycklig stopplinje eller parkeringsficka med eventuella hinder.}

    \actcat{Styrmodul}
    \activity{ctrl:drive}{Driva bilen}{5}{\aref{all:install,all:sw}}{Reglera
    taxins fart och taxins svängradie genom att kontrollera drivmotorn och
    svängmotorn utifrån kommunikationsmodulens kommandon.}

    \actcat{Sensormodul}
    \activity{sens:filter}{Bearbeta
    sensorvärden}{10}{\aref{all:sw,all:install}}{Filtrera brus från analoga
    sensorer och konvertera värden till SI-enheter.}
    \activity{sens:send}{Skicka värden}{10}{\aref{all:sw,all:install}}{Skicka
    värden till kommunikationsmodulen.}

    \actcat{Användargränssnitt}
    \activity{ui:rc}{Styra taxi}{10}{\aref{all:sw}}{Implementera gränssnitt för
    att styra taxin.}
    \activity{ui:param}{Inmatning}{5}{\aref{all:sw}}{Gränssnitt för att mata in
    konstantparametrer och en karta under körning.}
    \activity{ui:data}{Visa mätdata och karta}{5}{\aref{all:sw}}{Visa upp
    mätdata från taxin samt rita ut en karta över banan med taxins nuvarande
    position.}

    \actcat{Anslutningar}
    \activity{io:wireless}{Trådlös
    länk}{10}{\aref{all:sw,all:install}}{Implementera en trådlös länk mellan
    kommunikationsmodulen och fjärrklienten.}
    \activity{io:comm-sens}{Komm-sensor}{15}{\aref{all:sw,all:install}}{Implementera
    seriellt protokoll för att kommunicera mellan kommunikations- och
    sensormodul.}
    \activity{io:comm-styr}{Komm-styr}{15}{\aref{all:install,all:sw}}{Implementera
    seriellt protokoll för att kommunicera mellan kommunikations- och
    styrmodul.}

    \actcat{Testning}
    \activity{test:int}{Integrations\-testning}{10}{}{Utför integrationstest
    mellan de olika modulerna.}
    \activity{test:system}{Helsystemtestning}{30}{}{Utför systemtest av hela
    systemet.}

    \actcat{Dokument}
    \activity{doc:tecdoc}{Teknisk dokumentation}{50}{}{Dokumentera
    implementationen.}
    \activity{doc:study-pres}{Efterstudie och presentation}{20}{}{Utför en
    efterstudie och förbered en presentation.}
    \activity{doc:manual}{Användar\-manual}{20}{}{Skriv en användarmanual.}
    \activity{doc:status-time}{Tids- och statusrapporter}{20}{}{Skriva tids-
    och statusrapporter.}

    \actcat{Övrigt}
    \activity{impl:meet}{Möte}{40}{}{Genomföra möten.}
\end{activitylist}

\newpage

\end{document}
