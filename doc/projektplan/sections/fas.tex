\documentclass[projektplan/plan.tex]{subfiles}

\begin{document}
\section{Fasplan}
Fasplanen beskriver grovt de aktiviteter som ingår i de olika faserna och ger
en översikt på projektfaserna.

\subsection{Under projektet}
Projektet har delats upp i två faser; design och utförande.

\subsubsection*{Designfas}
Designfasen är planerad att börja vid BP2. Under designfasen ska produkten
designas och en designspecifikation ska sammanställas. Produktens
hårdvarudesign ska vara välspecificerad och ska beskrivas med blockdiagram,
kretsscheman och väl beskrivande text. När designspecifikationen är
färdigställd och godkänd avslutas designfasen vid BP3.

\subsubsection*{Utförandefas}
Uförandefasen påbörjas direkt efter designfasen vid BP3. Under utförandefasen
ska produkten konstrueras utefter designspecifikationen. Teknisk dokumentation
och användarhandledning ska också produceras. Produkten och tillhörande
dokument ska vara färdig enligt kravspecifikation vid BP5, därmed avslutas
utförandefasen.

\subsection{Efter projektet}
Vissa aktivitetar sker efter projektet och beskrivs i grova drag i det här
avsnittet.

\subsubsection*{Levererans}
Den färdiga produkten ska levereras till beställaren efter konstruktionens
slut. Ett acceptanstest görs med beställaren så att beställaren är nöjd med den
levererade produkten. En tekniskt dokumentation och en användarhandledning ska
skickas till beställaren. Hårdvaran ska återlämnas till ISY efter
konstruktionens slut.

\subsubsection*{Efterstudie}
Reflektion och återkoppling av projektet ska göras och lämnas in till
beställaren efter projektets slut. Analys av arbete och problem, samt vad
projektgruppen har uppnåt under projektets gång. En sammanfattning av viktiga
erfarenheter och goda råd till liknande projekt ska inkluderas i efterstudien.

\end{document}
