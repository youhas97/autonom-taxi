\documentclass[projektplan/plan.tex]{subfiles}
\begin{document}
\section{Dokumentplan}
{\renewcommand{\arraystretch}{1.6}
\begin{longtable}{p{4.5cm}p{1.5cm}p{5cm}p{2.2cm}p{1.2cm}}
    \bfseries Dokument &
    \bfseries Ansvarig &
    \bfseries Syfte &
    \bfseries Målgrupp &
    \bfseries Färdigdatum \\\hline
    Teknisk dokumentation &
    - &
    Ge konstruktionsunderlag, samt dokumentation för underhåll och
    felsökning. &
    Beställare &
    -
    \\
    Användarhandledning &
    - &
    Ge beskrivning av produktens användning. &
    Beställare, användare &
    -
    \\
    Designspecifikation &
    - &
    Beskriva implementeringen av hårdvaran samt mjukvaran till systemet. &
    Underhållare, utvecklare &
    -
    \\
    Projektplan &
    - &
    Beskriva hur projektet kommer att utföras och vilka strategier som
    tillämpas &
    Beställare &
    -
    \\
    Tidplan &
    - &
    Planera hur arbetet ska fördelas och när det ska utföras. &
    Löne\-avdelning, projekt\-ledaren &
    -
    \\
    Systemskiss &
    - &
    En skiss över implementationen av systemet &
    Beställare, utvecklare &
    -
    \\
    
    \endhead
\end{longtable}}


\section{Utvecklingsmetodik} 


\section{Utbildningsplan}
Eftersom gruppen inte har full kunskap om alla områden som behövs inom projektet behöver medlemmarna lära sig om olika funktioner och program. Inlärningen kommer ta tid och måste därför planeras och effektiviseras. Det som gruppen kommer behöva lära sig är följande:

\begin{labeling}{långt namn mer}
    \item[Open CV] Förstå hur OpenCV fungerar.
    \item[AVR] Förstå arkitektur och I/O för
    AVR-processorer.
    \item[Protokoll] Undersök protokoll som
    I$^2$C, SPI, UART m.m.
    \item[KiCad] Lära sig använda KiCad.
    \item[Mätinstrument] Kunna använda de mätintrument som krävs.
\end{labeling}

För att gruppledlemmarna ska kunna hålla allt de lär sig färskt i minnet ska inlärningen ske så nära intill motsvarande aktivitet som möjligt. Då undviker man att lägga onödig tid på repetition och kan istället använda den tiden till effektivtarbete. Tid för inlärningen står i tidsplanen.

\section{Rapporteringsplan}
För att både utvecklare och beställare ska ha en bra uppfattning om hur projektet ligger till ska det med jämna intervall skrivas tidsrapporter. Dessa skrivs främst av projektledaren givet att gruppmedlemmarna lämnar en personlig beskrivning av sitt arbete tillsammans med tidsrapport som underlag. Rapporterna ska skrivas varje veckoslut och ska innehålla alla den gångna veckans aktiviteter. Den slutgiltiga tidsrapporten ska sedan skickas till beställaren som enkelt kan se hur utvecklarna har jobbat och hur tiden har fördelats. Tider för när tidsrapporterna ska skrivas hittas i tidsplanen.
Det ska också skrivas statusrapporter. Dessa ska lämnas på begäran av beställaren och ska skrivas av projektledaren. Statusrapporterna ska ge en överblick av projektets nuvarande skede och hjälper beställaren att se hur långt projektet har kommit. Tidsplanering för statusrapporternafinns i tidsplanen.
\section{Mötesplan}
Gruppen kommer mestadels att arbeta i mindre grupper på separata platser under projekttiden. Olika medlemmar kommer arbeta på olika moduler parallellt med varandra. För att hålla kolla på hur alla ligger till och för att ha bra överblick över hur projektet ligger till tidsenligt så behöver gruppen träffas i möten. Möten kommer att hållas en gång varannan vecka på minst 40-60 minuter. Då mötena hålls tätt blir det lättare att se om projektet går i en annan riktning än vad som är tänkt, vilket gör att gruppen minimerar tiden som går åt till att korrigera arbetet. På mötena kommer gruppmedlemmarna att rapportera för varandra vad de har gjort, vad som gått bra och vad som gått dåligt. Det är också ett bra tillfälle att diskutera om något behöver justeras i tidsplanen eller om något krav behöver omförhandlas. För detaljerad bild av tidsplaneringen av mötena se tidsplanen.
\section{Resursplan}
\subsection{Personer}
\subsection{Material}
\subsection{Lokaler}
\subsection{Ekonomi}
\end{document}
