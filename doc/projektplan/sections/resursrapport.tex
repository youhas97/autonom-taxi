\documentclass[projektplan/plan.tex]{subfiles}
\begin{document}

\section{Dokumentplan}
Inför projektet ska planering och upplägg samt en skiss redovisas i en
projektplan, tidsplan och en systemskiss. Under projektet skall designen
specificeras i en designspecifikation. Teknisk dokumentation och användarmanual
för den slutgiltiga produkten skall också skrivas. Nedan visas varje dokument,
dess syfte och ett senast färdigdatum.
{\renewcommand{\arraystretch}{1.6}
\begin{longtable}{p{2.3cm}p{1.8cm}p{4.7cm}p{3.4cm}p{2.0cm}}
    \bfseries Dokument &
    \bfseries Ansvarig &
    \bfseries Syfte &
    \bfseries Distribueras till &
    \bfseries Färdig \\\hline
    Projektplan &
    \VARdokumentansvarig &
    Beskriva hur projektet kommer att utföras och vilka strategier som
    tillämpas. &
    Beställare &
    2018-10-04
    \\
    Tidsplan &
    \VARdokumentansvarig &
    Planera hur arbetet ska fördelas och när det ska utföras. &
    Beställare, Löne\-avdelning, projekt\-ledaren &
    2018-10-04
    \\
    Systemskiss &
    \VARdokumentansvarig &
    En skiss över implementationen av systemet. &
    Beställare, utvecklare &
    2018-10-04
    \\
    Design\-specifikation &
    \VARdokumentansvarig &
    Beskriva implementeringen av hårdvaran samt mjukvaran till systemet. &
    Underhållare, utvecklare &
    2018-11-09
    \\
    Teknisk dokumentation &
    \VARdokumentansvarig &
    Ge konstruktionsunderlag, samt dokumentation för underhåll och
    felsökning. &
    Beställare &
    2018-12-13
    \\
    Användar\-handledning &
    \VARdokumentansvarig &
    Ge beskrivning av produktens användning. &
    Beställare, användare &
    2018-12-13
    \\
    Efterstudie &
    \VARdokumentansvarig &
    Sammanfatta erfaranheterna från projektarbetet. &
    Beställare &
    2018-12-20
    \\
\end{longtable}}

\newpage
\section{Utvecklingsmetodik}
Grundmetodiken för projektet är att följa LIPS-modellen. Det är den mest
centrala delen för arbetets tillvägagångssätt. Utöver det ska gruppen dela upp
ansvaret för olika arbetsområden.

\subsection{Uppdelning}
Projektgruppen kommer mestadels att jobba i mindre delgrupper. Alla delgrupper
kommer ha ansvar för olika moduler. Eftersom det är främst tre moduler som ska
konstrueras kommer gruppen delas in i tre delgrupper. Delgrupperna kommer
bildas utefter vilken kunskap medlemmarna har om respektive modul för att
underlätta arbetet.

\subsection{Programmeringsspråk}
Olika språk ska användas för olika områden i systemet. Kommandon och
gränssnittet för den bärbara datorn kan förslagsvis skrivas i Python medan
mikrokontrollerna och alla behandlingar av sensordata kan förslagsvis skrivas i
C.

\newpage
\section{Utbildningsplan}
\noindent
Eftersom gruppen inte har full kunskap om alla områden som behövs inom
projektet behöver medlemmarna lära sig om olika funktioner och program.
Inlärningen kommer ta tid och måste därför planeras och effektiviseras. Det som
gruppen kommer behöva lära sig är följande:
\begin{labeling}{långt namn mer}
    \item[Open CV] Förstå hur OpenCV fungerar.
    \item[AVR] Förstå arkitektur och I/O för
    AVR-processorer.
    \item[Protokoll] Undersök protokoll som
    I$^2$C, SPI, UART m.m.
    \item[KiCad] Lära sig använda KiCad.
    \item[Mätinstrument] Kunna använda de mätintrument som krävs.
\end{labeling}
För att gruppledlemmarna ska kunna hålla allt de lär sig färskt i minnet ska
inlärningen ske så nära intill motsvarande aktivitet som möjligt. Då undviker
man att lägga onödig tid på repetition och kan istället använda den tiden till
effektivt arbete. Tid för inlärningen står i tidsplanen.

\section{Rapporteringsplan}
För att både utvecklare och beställare ska ha en bra uppfattning om hur
projektet ligger till ska det med jämna intervall skrivas tidsrapporter. Dessa
skrivs främst av projektledaren givet att gruppmedlemmarna lämnar en personlig
beskrivning av sitt arbete tillsammans med tidsrapport som underlag.
Rapporterna ska skrivas varje veckoslut och ska innehålla alla aktiviteter från
den gångna veckans arbete, detta sker vid gruppmöten som förekommer en gång i
veckan. Den slutgiltiga tidsrapporten ska sedan skickas till beställaren som
enkelt kan se hur utvecklarna har jobbat och hur tiden har fördelats. Tider för
när tidsrapporterna ska skrivas hittas i tidsplanen. Det ska också skrivas
statusrapporter. Dessa ska lämnas på begäran av beställaren och ska skrivas av
projektledaren. Statusrapporterna ska ge en överblick av projektets nuvarande
skede och hjälper beställaren att se hur långt projektet har kommit.
Tidsplanering för statusrapporterna finns i tidsplanen.

\section{Mötesplan}
Gruppen kommer mestadels att arbeta i mindre grupper på separata platser under
projekttiden. Olika medlemmar kommer arbeta på olika moduler parallellt med
varandra. För att hålla kolla på hur alla ligger till och för att ha bra
överblick över hur projektet ligger till tidsenligt så behöver gruppen träffas
i möten. Möten kommer att hållas en gång varje vecka på cirka 40 minuter. Då
mötena hålls tätt blir det lättare att se om projektet går i en annan riktning
än vad som är tänkt, vilket gör att gruppen minimerar tiden som går åt till att
korrigera problem i arbetet. På mötena kommer gruppmedlemmarna att tillsammans
gå summera föregående veckas jobb som underlag för projektledarens
sammanställning av en tidsrapport. Under mötet skall även nästkommande vecka
planeras utefter den nuvarande statusen av projektet. Om föregående veckas
arbete ej överensstämmer med tidsplanen skall tidsplanen uppdateras under
gruppmötet och skickas till beställaren tillsammans med tidsrapporten.

\section{Resursplan}
Det här avsnittet innehåller information om resurser inom projektet.

\subsection{Personer}
Gruppen består utav 7 personer som totalt ska arbeta 1120 timmar med projektet
efter beslutspunkt 2.

\begin{itemize}
    \item Jakob Arvidsson
    \item Juan Basaez
    \item Johan Can
    \item Dennis Dereichei
    \item Yousef Hashem
    \item Emir Hadzisalihovic
    \item Noah Hellman

\end{itemize}
Gruppen har bestämt att alla ska jobba samma antal timmar under projektet, det
vill säga att varje gruppmedlem ska jobba 160 timmar.

En handledare ska finnas tillgänglig max 2 timmar per vecka under
projektprocessen. Projektgruppen ska ha möjligheten att kontakta handledaren
vid behov. Dessutom kan gruppen hänvisas till en expert om handledaren anser
det vara nödvändigt.

\subsection{Material}
Även om designen av konstruktionen inte än har bestämts vet man redan att
projektgruppen kommer använda sig av en Raspberry Pi, sensorer, processorer, en
kamera, en bärbar dator, bilens chassi samt mätutrustning. Allt som ska
behövas finns tillgängligt på universitetet, men projektgruppen kan beställa
komponenter från externa beställare om det bestämts och anses vara nödvändigt.

\subsection{Lokaler}
Projektetgruppen ska under utförandeprocessen, räkna med tillgång till
labbsalen Muxen (där mätutrustning finns), olika mötessalar och universitets
utrymmen vid testning av bilen.

\subsection{Ekonomi}
Projektet har en budget på 1120 timmar.

\end{document}
