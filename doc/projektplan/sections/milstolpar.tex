\documentclass[projektplan/plan.tex]{subfiles}
\begin{document}
\section{Milstolpar och beslutspunkter}
\subsection{Milstolpar}
\subsection{Beslutspunkter}
En beslutspunkt är ett moment där projektgruppen har ett kort möte med
beställaren. Projektledaren ser till att de dokument som är fundamenten för
respektive beslut är färdiga och granskade. Dokumenten ska vara distribuerade
till vederbörande parter i god tid innan mötet. På mötet följs mallen för
beslutspunkten och fungerar som en checklista. Mallen kan sedan sparas som ett
beslutsprotokoll.

\subsubsection*{BP0} Vid beslutspunkt 0 godkänns projektdirektivet vilket t.ex. innebär att
projektet ska göras, medel ska tillelas projektgruppen för att de förstudier
som ska göras till  BP1 blir utförda.

\subsubsection*{BP1} 
Vid beslutspunkt 1 har förstudierna utförts och det som ska göras är specificerat. Det
finns också en stadig plan för hur projektet skall fortsätta, inte minst fram
till BP2. Beställaren bestämmer i BP1 om projektgruppen kommer att fortsätta
med förberedelser inför utförandefasen. Här ska kravspecifikationen samt en
preliminär projektplan vara färdigt.

\subsubsection*{BP2}
Vid beslutspunkt 2 skall beslut tas om att påbörja projektets utförandefas. Detta är ett
essentiellt beslut då projektet fortsättningsvis kommer att kosta pengar/tid.
Många involveras och de komponenter som krävs för att utföra projekter kommer
att köpas in. De dokument som ska finnas vid BP2 är en godkänd
kravspecifikation, systemskiss, projektplan och ett kundkontrakt.

\subsubsection*{BP3}
Vid beslutspunkt 3 ska en mer noggrann uppskattning av arbetet i projektet
göras. Beslutspunkten hålls efter designfasen då man specificerat exakt hur
alla moment ska geomföras och hur allt ska konstrueras. Dokumenten som ska vara
färdiga är designspecifikationen och eventuellt en modifierad projektplan.

\subsubsection*{BP4}
Vid beslutspunkt 4 genomförs extra genomgång av utförandefasen (behövs inte i
detta projekt). Produkten och
arbetets resultat/kvalitet granskas. Beslutspunkten är placerad så att ett
antal delmoment har uppnåtts och testats. Här ska designspecifikationen och
testprotokoll vara färdigt.

\subsubsection*{BP5}
Vid beslutspunkt 5 avgörs om projektet har nått målen och om resultatet ska
användas.Arbetet skall vara utfört som förväntat, produkten ska hålla rätt
kvalitet, alla dokument ska finnas. Här ska leveransen kontrolleras för att man
ska kunna avgöra om den är förberedd för leverans.

\subsubsection*{BP6}
Vid beslutspunkt 6 beslutar man om projektet ska avslutas. Här avgörs om
projektgruppen ska uppösas och om resuserna ska lämnas tillbaka. Här godkänns
också leveransen.

\begin{center}
    \begin{tabular}{| l | l | l |}
    0 & Godkännande av projektdirektivet, beslutande av förstudier & 2018??? 
    \end{tabular}
\end{center}    


\section{Tidplan}

\section{Förändringsplan}

\section{Kvalitetsplan}
För att bibehålla en hög kvalitet ska material såsom kod, dokument, presentationer etc. granskas av någon eller några av medlemmarna i gruppen.
Även testning av mjukvara och hårdvara ska utföras.

\subsection{Granskningar}
Allt material som produceras av gruppen ska granskas av samtliga medlemmar. Detta för att andra medlemmar kan ha åsikter på förbättringar som författaren inte tänkt på eller inte har kunskap om. På så sätt fås en högre kvalitet. I första hand ska den medlem som har mest kunskap inom det område granska.

\subsubsection{Kod}	
Granskning av kod kan innefatta optimering såsom minska redundans samt effektivisera. Det kan innefatta att skriva om så att kod kan skrivas på ett kortare sätt eller med modulärt.

\subsubsection{Dokument}	
Genom att granska dokument kan fel såsom stavfel, formuleringar och meningsuppbyggnader rättas till. Detta är speciellt önskvärt då en utomstående kan upptäcka fel som författaren kan ha svårt att hitta.

\subsubsection{Presentationer}	
Presentationer granskas för att säkerhetsställa att inga felaktigheter i fakta, utförande eller ambitioner presenteras.

\subsubsection{Schema}	
Schemat ska granskas av alla medlemmar för att kunna anpassas efter allas scheman. Detta för att se till att så många som möjligt kan närvara vid samtliga tillfällen. 

\subsection{Testplan}
För att testa mjukvaran ska testprogram skrivas som matar in indata och förväntar sig en viss utdata. Skiljer sig förväntad utdata från faktiskt data ses detta som ett fel. Varje modul kommer testas enskilt för att sedan testa kommunikation mellan dessa. Testning av hårdvara sker genom att se till att de olika hårdvarukomponenter kan kommunicera sinsemellan samt att de beter sig som annonserad.

När både mjuk- och hårdvara har testats och det har verifieras att dessa fungerar separat ska de testas tillsammans. Tills konstruktionen har färdigställts går det inte har sätta upp när dessa tester ska utföras men testerna kan ändå definieras.

Testar som ska utföras:
•	Röra sig korrekt när den detekterar ett streck i en viss färg.
•	Parkera i en parkeringsficka.
•	Kan åka framåt, bakåt och åt båda sidorna.
•	Kartan kan presenteras korrekt utifrån den data som läses in.
•	Åka runt banan utan att stanna.
•	Hålla sig inom en fil.
•	Undvika hinder.
•	Testa kommunikation mellan modulerna.
•	Få kontakt med bilen från datorn.

\section{Riskanalys}
fill me

\section{Prioriteringar}
fill me

\section{Projektavslut}
fill me

\end{document}
