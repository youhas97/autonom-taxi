\documentclass[usermanual/man.tex]{subfiles}
\section{Karta}
Beskriv hur kartan är representerad, förklara vad noderna betyder. Visa hur man
sparar karta och laddar karta.

Kartan kommer att representeras av noder av olika färger som definierar stop,
infart till rondell, utfart till rondell och parkeringsficka. Det finns även
tomma noder som endast finns till för att ge en kartan en bättre
visuell representation. Färgerna för respektive nodtyp är röd för stop, gul för
parkering, mörkblå för infart till rondell, ljusblå för utfart till rondell och
vit för den tomma noden.

För att skapa kartan högerklickar man på den stora rutan längst till höger där
alternativen create och delete hittas. Genom att välja create kan
respektive nodtyp placeras ut vid platsen där man högerklickade. Man väljer en
nod genom att vänsterklicka på den, färgen på noden blir då grön för att visa
att noden är vald. Om en nod skall tas bort vid en eventuell felplacering eller byte av kartdesign, väljer
man den noden som skall tas bort genom att vänsterklicka på den, för att sedan
högerklicka och välja alternativet delete.
