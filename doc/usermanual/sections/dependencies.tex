\documentclass[usermanual/man.tex]{subfiles}
\section{Väsentligheter}
Apparaten är till större delen utvecklad i en linux miljö. Man behöver
installera ett antal paket för att kunna programmera mikrokontrollern samt
kompilera mjukvaran i linuxmiljö. Flashning av mikroprocessorerna kan göras i
windows med hjälp av \mono{Atmel Studio}.

\subsection{Programvara}
\subsubsection{OpenCV2}
OpenCV version 3.4.3 är installerad på raspberry pi modulen.
\subsubsection{avrdude}
Avrdude behövs för flashning av respektive AVR.
\subsubsection{Python3}
Python version 3.6 eller nyare krävs för att köra det grafiska användargränssnittet.
\subsubsection{Tkinter}
Tkinter är ett standardbibliotek för grafiska användargränssnitt som krävs.

\subsection{Hårdvara}
För att kunna använda bilen krävs ett laddat batteri. Laddning av batteri sker
med avsedd laddare för batteriet med en ström på max 3A.
En dator med WiFi kompabilitet är väsentligt för att kunna använda bilen.

\subsection{Linux}
Då apparaten är utvecklad i en linux miljö rekommenderas att all kompilering
och flashning sker i denna miljö. För att kunna göra allt detta krävs det att
man placerar sig i mappen \mono{tsea29-taxi/src}. När alla instruktioner ovan
har följts kan apparaten börja användas.
